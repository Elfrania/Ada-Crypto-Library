\chapter{Crypto.Random}
\section{Einf�hrung}
Dieses Paket greift defaultm��ig auf ``dev/random/'' zu.
Unter Linux ist \\``/dev/random'' ein Pseudodevice das kryptographisch sichere
Pseudozufallsbits liefert. Wenn sie ein Betriebsystem verwenden das kein 
solches Pseudodevice besitzt, dann k�nnen sie dies mit Hilfe von mit 
Brian Warners ``Entropy Gathering Daemon''(EGD), der unter
\textit{http://www.lothar.com/tech/crypto/} zu finden 
ist,  nachinstallieren. Der EGD ist ein Userspace-Implementierung des 
Linux-Kernel-Devices ``dev/random/'', das kryptograpisch sichere
Zuallsbits generiert.\\
Im privaten Teil befindet sich die Variable ``Path''. Diese Variable gibt den
Pfad zu einer Datei, Named Pipe oder Ger�t das krytograpisch sichere
Zufallsbits enth�lt an.  Dadurch sind Sie in der Lage die ACL auch an eine
kryptograpisch sichere (Pseudo-)Zufallsbitquelle eines nicht POSIX kompatiblen
Betriebsystems wie Windows XP anbinden.\\
Weiterhin besteht die M�glichkeit unter Windows den Marsaglia- RNG zu benutzen.
Er ist ein lagged- Fibonacci Sequenz Generator und erzeugt 24-Bit real numbers
im Intervall [0,1].\\
Eigenschaften: Kombination zweier unterschiedlicher Generatoren, welche fort-
laufende Bits erzeugen. Diese werden durch zwei weitere Generatoren zu einer 
Tabelle Kombiniert.
Die Prozedur ''Start'' initialisiert diese Tabelle, welche von der Funktion 
''Next'' genutzt wird um die n�chste Zufallszahl zu generieren.

\section{API}
\subsection{Prozeduren}
\begin{tabular}{p{\textwidth}}
  \begin{lstlisting}{}
    procedure Read(B : out Byte);
    procedure Read(W : out Word);
    procedure Read(D : out DWord);
  \end{lstlisting}
  Bei diesen drei Prozeduren wird B, W oder D mit Bits aus der
  verwendeten Pseudozufallsbit-Quelle gef�llt.\\ \ \\
  \hline\\
\end{tabular}

\begin{tabular}{p{\textwidth}}
  \begin{lstlisting}{}
    procedure Read(Byte_Array  : out Bytes);
    procedure Read(Word_Array  : out Words);
    procedure Read(DWord_Array : out DWords);
  \end{lstlisting}\\
  Diese drei Prozeduren wirden die �bergebenen Arrays mit Bits aus der 
  verwendeten Pseudozufallsbit-Quelle gef�llt. \\ \ \\
\end{tabular}

\begin{tabular}{p{\textwidth}}
  \begin{lstlisting}{}
    procedure Start(
    New_I : Seed_Range_1 := Default_I;
    New_J : Seed_Range_1 := Default_J;
    New_K : Seed_Range_1 := Default_K;
    New_L : Seed_Range_2 := Default_L);
  \end{lstlisting}\\
  \\ \ \\
\end{tabular}

\subsection{Exceptions}
\begin{tabular}{p{\textwidth}}
  \begin{lstlisting}{}
    Random_Source_Does_Not_Exist_Error : exception;
  \end{lstlisting}\\
  Diese Ausnahme wird aufgerufen, wenn kein (kryptograpisch sicherer) 
  Pseudozufallsbit-Quelle gefunden wurde.\\ \ \\
  \hline\\
  \begin{lstlisting}{}
    Random_Source_Read_Error : exception;
  \end{lstlisting}\\
  Wenn ein Lesefehler bei einem Zugriff auf die Pseudozufallsbit-Quelle 
  auftritt, wird die folgende Aufnahme ausgeworfen. \\ \ \\
  \hline\\
\end{tabular}

\begin{lstlisting}{}
\end{lstlisting}
