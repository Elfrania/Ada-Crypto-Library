\chapter{Crypto.Hashfunction}\label{hash}
Mit Hilfe dieses generischen Paketes l�sst sich aus dem Algorithmus einer
kryptographsichen Hashfunktion (siehe Kapitel \ref{algoh})
eine krypto. Hashfunktion erstellen die sich hervorragend f�r folgende Zwecke 
einsetzen:
\begin{itemize}
\item Integrit�ts�berpr�fung von Nachrichten
\item Generierung und Verifizierung von digitalen Signaturen.
\item Generierung von Zufallszahlen bzw. Zufallsbits
\end{itemize}
Der Zweck dieses Paketes ist es die API f�r Hashfunktionen zu vereintheilichen
und zu vereinfachen. Aus Sicht der Sicherheit spricht hier nichst dagegen 
direkt die native API aus \texttt{Crypto.Symmetric.Algorithm} zu verwenden.\\

%%%%%%%%%%%%%%%%%%%%%%%%%%%%%%%%%%%%%%%%%%%%%%%%%%%%%%%%%%%%%%%%%%%%%%%%%%%%
%%%%%%%%%%%%%%%%%%%%%%%%%%%%%%%%%%%%%%%%%%%%%%%%%%%%%%%%%%%%%%%%%%%%%%%%%%%%

\section{Generischer Teil}
 \begin{lstlisting}{}
generic
   type Hash_Type                 is private;
   type Message_Type              is private;
   type Message_Block_Length_Type is private;
   
   with procedure Init(Hash_Value : out Hash_Type) is <>;

   with procedure Round(Message_Block : in     Message_Type;
                        Hash_Value    : in out Hash_Type) is <>;

   with function Final_Round(Last_Message_Block  : Message_Type;
                             Last_Message_Length :
                             Message_Block_Length_Type;
                             Hash_Value          : Hash_Type)
                             return Hash_Type is <>;

   with procedure Hash(Message    : in Bytes;
                       Hash_Value : out  Hash_Type) is <>;

   with procedure Hash(Message    : in String;
                       Hash_Value : out  Hash_Type) is <>;

   with procedure F_Hash(Filename : in String;
                         Hash_Value : out  Hash_Type) is <>;

 \end{lstlisting}\ \\

%%%%%%%%%%%%%%%%%%%%%%%%%%%%%%%%%%%%%%%%%%%%%%%%%%%%%%%%%%%%%%%%%%%%%%%%%%%
%%%%%%%%%%%%%%%%%%%%%%%%%%%%%%%%%%%%%%%%%%%%%%%%%%%%%%%%%%%%%%%%%%%%%%%%%%%

\section{API}
Die API einer generischen Hashfunktion besteht aus einer High- und einer 
Low-Level-API. Die Low-Level-API sollen sie nur verwenden, wenn sie sich mit
krptographischen Hashfunktionen auskennen. Wenn dies nicht der Fall ist, dann
verwenden sie bitte die folgende High-Level-API.

\subsection{High-Level-API}
\begin{lstlisting}{}
  function Hash  (Message  : Bytes)  return Hash_Type;
  function Hash  (Message  : String) return Hash_Type;
  
  function F_Hash(Filename : String) return Hash_Type;
\end{lstlisting}
Die Funktion \textbf{Hash} liefert den Hashwert einer Nachricht 
(\textit{Message}). Bei der Nachricht kann es sich dabei entweder um ein 
Byte-Array oder einen String handeln. Die Funktion \textbf{F\_Hash} gibt den 
Hashwert der Datei \textit{Filename} zur�ck.  Beispielsweise liefert 
die Codezeile
\begin{lstlisting}{}
  H := F_Hash("/bin/ls")
\end{lstlisting}
den Hashwert von \texttt{/bin/ls}.\\ \ \\


\subsection{Low-Level-API}
Die Low-Level-API besteht aus einer Funktion und zwei Prozeduren. 

\begin{itemize}
\item Die Prozedur \textbf{Init} initialisiert bzw. reinitailisert die 
  Hashfunktion. Jedesmal wenn eine Nachricht gehashed werde soll muss 
  zun�chst diese Prozedure aufgerufen werden.
  \begin{lstlisting}{}
 procedure Init;
  \end{lstlisting}

\item Mit der Prozedure \textbf{Round}, k�nnen iterativ Nachrichtenbl�cke
  gehashed werden. 
  \begin{lstlisting}{}
procedure Round(Message_Block : in Message_Type);
 \end{lstlisting}
 
\item Die Funktion \textbf{Final\_Round} padded und hashed anschlie�end einen
  Nachrichtenblock \textit{Last\_Message\_Block}. Auf Grund des Paddings muss
  die Bytel�nge des Nachrichtenmaterials \textit{Last\_Message\_Length} 
  angegeben werden.  Denn eine Nachricht ist i.d.R.  k�rzer als eine 
  Nachrichtenblock vom Typ \textit{Message\_Type}. Der R�ckgabewert dieser 
  Funktion entspricht dem finalen Hashwert einer Nachricht.
   \begin{lstlisting}{}
function Final_Round(Last_Message_Block  : Message_Type;
                     Last_Message_Length : Message_Block_Length_Type)
                     return Hash_Type;
  \end{lstlisting}
\end{itemize}\ \\

%%%%%%%%%%%%%%%%%%%%%%%%%%%%%%%%%%%%%%%%%%%%%%%%%%%%%%%%%%%%%%%%%%%%%%%%%%%
%%%%%%%%%%%%%%%%%%%%%%%%%%%%%%%%%%%%%%%%%%%%%%%%%%%%%%%%%%%%%%%%%%%%%%%%%%%
%%%%%%%%%%%%%%%%%%%%%%%%%%%%%%%%%%%%%%%%%%%%%%%%%%%%%%%%%%%%%%%%%%%%%%%%%%%

\section{Anwendungsbeispiel}
\subsection{High-Level-API}
Das folgende Beispiel gibt den SHA-256 Hashwert von \textbf{/bin/ls} aus.
 \begin{lstlisting}{}
with Ada.Text_IO;
with Crypto.Types,
with Crypto.Hashfunction_SHA256;


use Crypto.Types, Ada.Text_IO;

procedure example is
   package SHA256 renames Crypto.Hashfunction_SHA256;

   Hash : W_Block256 := SHA256.F_Hash("/bin/ls");
begin
   for I in Hash'Range loop
      Put(To_Hex(Hash(I)));
   end loop;
   Put_Line(" /bin/ls");
end example;
 \end{lstlisting}

\pagebreak

\subsection{Low-Level-API}
\begin{lstlisting}{}
with Ada.Text_IO;
with Crypto.Types;
with Crypto.Hashfunction;
with Crypto.Symmetric.Algorithm.SHA256;

use Ada.Text_IO;
use Crypto.Types;
use Crypto.Symmetric.Algorithm.SHA256;

pragma Elaborate_All (Crypto.Hashfunction);

procedure Example_Hashing is
   package WIO is new  Ada.Text_IO.Modular_IO(Word);

    package SHA256 is
      new Crypto.Hashfunction(Hash_Type    => W_Block256,
                              Message_Type => W_Block512,
                              Message_Block_Length_Type =>
                              Crypto.Types.Message_Block_Length512);

   Message : String :=  "All your base are belong to us!";

   W : Words := To_Words(To_Bytes(Message));

   M : W_Block512 := (others => 0);
   H : W_Block256;
begin
   M(W'Range) := W;

   SHA256.Init;

   -- Berechne den Hashwert von Message
   H := SHA256.Final_Round(M, Message'Last);

   -- Gib den Hashwert von Message aus
   for I in W_Block256'Range loop
      WIO.Put(H(I), Base => 16);
      New_line;
   end loop;
   New_Line;
end Example_Hashing;
\end{lstlisting}

\pagebreak

\section{Anmerkung}
Sie m�ssen nicht jedes mal von neuem aus einem daf�r vorgesehenen symmetrischen
Algorithmus eine Hashfunktion generieren. Stattdessen k�nnen Sie auch eine der 
vorgefertigten Hashfunktionen verwenden.
\begin{itemize}
\item Crypto.Hashfunktion\_SHA1
\item Crypto.Hashfunktion\_SHA256
\item Crypto.Hashfunktion\_SHA512
\item Crypto.Hashfunktion\_Whirlpool
\end{itemize}


