\section{Bin�re K�rper GF($2^m$)}
\subsection{Erl�uterungen}
In dem separatem Body Crypto.Types.Big\_Numbers.Mod\_Binfield\_Utils befinden 
sich Funktionen und Prozeduren bei denen eine \texttt{Big\_Unsigned} Variable 
als Element aus dem K�rper GF($2^m$) interpretiert wird.\\ \ \\
\textbf{Beispiel:}
\begin{lstlisting}[frame=tlrb]{}
  -- Es wird das 1., 2. und 6. Bit von A gesetzt.
  -- Danach enspricht A dem Element/Polynom z^5 + z + 1
  A := Big_Unsigned_One;
  A := A xor Shift_left(Big_Unsigned_One,1);
  A := A xor Shift_left(Big_Unsigned_One,5);
\end{lstlisting}
Die \texttt{Big\_Unsigned} Variable F ist im folgenden immer ein irreduzible 
Polynom $f(z)$ vom Grad m.
Der Zugriff erfolgt �ber das Pr�fix \textbf{Binfield\_Utils.}\\

%%%%%%%%%%%%%%%%%%%%%%%%%%%%%%%%%%%%%%%%%%%%%%%%%%%%%%%%%%%%%%%%%%%%%%%%%%%
\subsection{API}

\begin{tabular}{p{\textwidth}}
\begin{lstlisting}{}
function B_Add(Left,Right : Big_Unsigned) return Big_Unsigned;
function B_Sub(Left,Right : Big_Unsigned) return Big_Unsigned;
\end{lstlisting}\\
Diese beiden Funktion berechnen Left xor Right.
Dies entspricht einer Addtion bzw. Subtraktion in GF($2^m$).\\ \ \\
\hline
\end{tabular}

%%%%%%%%%%%%%%%%%%%%%%%%%%%%%%%%%%%%%%%%%%%%%%%%%%%%%%%%%%%%%%%%%%%%%%%%%%%


\begin{tabular}{p{\textwidth}}
\begin{lstlisting}{}
  function B_Mult(Left, Right, F : Big_Unsigned) return Big_Unsigned;
\end{lstlisting}\\
Diese Funktion berechnet Left $\cdot$ Right mod F.\\ \ \\
\hline
\end{tabular}

%%%%%%%%%%%%%%%%%%%%%%%%%%%%%%%%%%%%%%%%%%%%%%%%%%%%%%%%%%%%%%%%%%%%%%%%%%%

\begin{tabular}{p{\textwidth}}
  \begin{lstlisting}{}
function B_Square(A, F : Big_Unsigned)    return Big_Unsigned;
  \end{lstlisting}\\
  Diese Funktion berechnet $A^2\; \bmod\; F$.\\ \ \\
  \hline
\end{tabular}

%%%%%%%%%%%%%%%%%%%%%%%%%%%%%%%%%%%%%%%%%%%%%%%%%%%%%%%%%%%%%%%%%%%%%%%%%%%

\begin{tabular}{p{\textwidth}}
\begin{lstlisting}{}
function B_Div (Left, Right, F : Big_Unsigned) return Big_Unsigned;
\end{lstlisting}\\
Diese Funktion berechnet Left/Right mod F.\\ \ \\
\hline
\end{tabular}

%%%%%%%%%%%%%%%%%%%%%%%%%%%%%%%%%%%%%%%%%%%%%%%%%%%%%%%%%%%%%%%%%%%%%%%%%%%

\begin{tabular}{p{\textwidth}}
  \begin{lstlisting}{}
function B_Mod(Left, Right  : Big_Unsigned) return Big_Unsigned;
  \end{lstlisting}\\
  Diese Funktion berechnet Left mod Right.\\ \ \\
  \hline
\end{tabular}


%%%%%%%%%%%%%%%%%%%%%%%%%%%%%%%%%%%%%%%%%%%%%%%%%%%%%%%%%%%%%%%%%%%%%%%%%%%


\begin{tabular}{p{\textwidth}}
  \begin{lstlisting}{}
function B_Inverse(X, F : Big_Unsigned) return Big_Unsigned;
  \end{lstlisting}\\
  Diese Funktion berechnet $X^-1\; \bmod\; F$.\\ \ \\
  \hline
\end{tabular}


%%%%%%%%%%%%%%%%%%%%%%%%%%%%%%%%%%%%%%%%%%%%%%%%%%%%%%%%%%%%%%%%%%%%%%%%%%%

\section{Exceptions}

\begin{lstlisting}{}
  Constraint_Size_Error : exception;
  Conversion_Error      : exception;
  Division_By_Zero      : exception;
  Is_Zero_Error         : exception;
\end{lstlisting}
