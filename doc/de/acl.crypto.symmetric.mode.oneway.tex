\chapter{Crypto.Symmmetric.Mode.Oneway}

Dieses generische Paket betreibt eine Einwegblockchiffre in einen bestimmten 
Einweg-(Betriebs-)modus. Dieser verkn�pft gew�hnlich eine Einweg-Blockchiffre 
mit einer R�ckkopplung und einigen einfachen Operationen (+, xor). Ein 
Einweg-Modus wird mit
Hilfe eines zuf�lligen �ffentlichen Startwert (Initial Value (IV)) 
initialisiert. Der Chiffretext ist damit nicht nur von der verwendeten Chiffre,
Klartext und Schl�ssel abh�ngig, sondern auch von dem zuf�lligen Startwert. 
Wenn Sie nun einen Klartext mehrmals mit der gleichen Chiffre und dem gleichen
Schl�ssel aber unterschiedlichen IVs verschl�sselt, erhalten Sie
unterschiedliche Chiffretext und durch die R�ckkopplung in einem Einweg-Modus
werden gleiche Klartextbl�cke zu unterschiedlichen Chiffretextbl�cken 
chiffriert, d.h ein Einweg-Betriebsmodus verschl�sselt zwei Klartextbl�cke
p1 und p2 mit p1=p2, mit  �berw�ltigender Wahrscheinlichkeit, zu zwei 
Chiffretextbl�cke c1 und c2 mit c1$\not=$c2. Damit ist es nun m�glich mehrere 
Nachrichten mit dem selben Schl�ssel zu verschl�sseln.\\
\textbf{Vorsicht: Um einen Chiffretext zu entschl�sseln ben�tigen Sie den 
gleichen Schl�ssel und Startwert wie bei der Verschl�sselung.}
Aus diesem Grund sollte der Startwert immer mit dem zugeh�rigen Chiffretext
aufbewahrt werden. \textbf{ Die Sicherheit eines Modus ist unabh�ngig vom 
\glqq Bekanntheitsgrad\grqq des Startwertes}. Daher Multipliziert man den
Startwert meist mit dem Chiffretext zu dem endg�ltigen Chiffrat indem man den
Startwert vor dem Chiffretext h�ngt ($IV*C = C'= IV||C$).\\

\section{Anmerkungen}
\begin{itemize}
\item 
  F�r die Einweg-Modus gilt das gleiche wie f�r einen normalen Modus.
  Falls ein normaler Modus auch als Einweg-Modus zur Verf�gung steht, dann
  sollte Sie diesen dem normalen Modus vorziehen, da dieser etwas schlanker
  ist. 
\item Die \textbf{API} ist identisch zu den normalen Modi [ \ref{mode-api} ]
\item Unterst�tze Modi
  \begin{itemize}
  \item Cipher-Feedback-Modus (CFB)  [ \ref{CFB} ]
  \item Counter-Modus (CTR) [ \ref{CTR} ]
  \item Output-Feedback-Modus (OFB) [ \ref{OFB} ]
  \end{itemize}
\end{itemize}

\section{Anwendungsbeispiel}
\begin{lstlisting}{}
with Crypto.Types, Ada.Text_IO, Crypto.Symmetric.Mode.Oneway_CTR;
with Crypto.Symmetric.Oneway_Blockcipher_Twofish128;

procedure Bsp_Oneway_Modus_CTR
   package TF128 renames Crypto.Symmetric.Oneway_Blockcipher_Twofish128;

   -- Benutze die Blockchiffre in einem sicheren Modus
   package Twofish128 is new Crypto.Symmetric.Mode.Oneway_CTR(TF128);

   use Ada.Text_IO, Crypto.Types, Twofish128;

   --Schluessel
   Key : B_Block128:=(16#2b#, 16#7e#, 16#15#, 16#16#, 16#28#, 16#Ae#,
                      16#D2#, 16#A6#, 16#Ab#, 16#F7#, 16#15#, 16#88#,
                      16#09#, 16#Cf#, 16#4f#, 16#3c#);

   --Startwert
   IV : B_Block128 := (15 => 1, others => 0);

   --Nachricht
   Message : String :="All your Base are belong to us! ";

   -- Nachrichtenbloecke
   P : array(1..2) of B_Block128 :=
     (To_Bytes(Message(1..16)), To_Bytes(Message(17..32)));

   --Chiffretextbloecke
   C : array(0..2) of B_Block128;

begin
   --Initialisierung
   Init(Key, IV);

   -- 1. Chiffreblock = Startwert.
   C(0) := IV;

    -- Verschluesselung
   for I in P'Range loop
      Encrypt(P(I), C(I));
   end loop;

   -- Fuer die Entschluesselung wird die Chiffre mit dem
   -- gleichen Startwert wie bei der Entschluesselung reinitalisiert
   Set_IV(C(0));

   -- Entschluesselung
   for I in P'Range loop
      Decrypt(C(I), P(I));
      Put(To_String(P(I)));
   end loop;

end Bsp_Oneway_Modus_CTR;
\end{lstlisting}




