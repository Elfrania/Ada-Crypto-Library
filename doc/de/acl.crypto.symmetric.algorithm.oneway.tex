\chapter{Crypto.Symmetric.Algorithm.Oneway}\label{onealg}

Jede symmetrische Chiffre und kryptographische Hashfunktion
hat ein Unterpaket \glqq Oneway \grqq. Bei diesem Paket handelt es sich um 
eine symmetrische Chiffre ohne Entschl�sslungsoperation.\\
Auch bei den Oneway-Algorithmen sollte sie davon Abstand nehmen die vorhanden
API direkt zu verwenden.\\ \ \\

\subsubsection{API}

Ein Oneway-Paket stellt dem Anwendungsprogrammierer folgende Typen und 
Operationen zur Verf�gung.


\begin{itemize} 
   \item Einen privaten Typ \textbf{Cipherkey\_Oneway} der deterministisch aus
     einem  Schl�ssel generiert wird. Mit dessen Hilfe  kann man 
     die  Verschl�sselungsfunktion des entsprechenden Paketes aufrufen.
   \item Eine Prozedur \textbf{Prepare\_Oneway\_Key} die aus einem Schl�ssel
     den zugeh�rigen Cipherkey erstellt.\\ \ \\
     \textit{Beispiel:}
     \begin{lstlisting}{}
procedure Prepare_Oneway_Key(Key       : in B_Block; 
                             Cipherkey : out Cipherkey_Oneway);
     \end{lstlisting}
     
   \item Eine Prozedur \textbf{Encrypt\_Oneway}. 
     Diese Prozedur verschl�sselte einen Klartext (Plaintext) mit Hilfe eines
     Cipherkeys in einen Chiffretext (Ciphertext).\\ \ \\
     \textit{Beispiel:}
     \begin{lstlisting}{}
 procedure Encrypt_Oneway(Cipherkey  : in  Cipherkey_Oneway;
                          Plaintext  : in  B_Block;
                          Ciphertext : out B_Block);
     \end{lstlisting}
\end{itemize}

