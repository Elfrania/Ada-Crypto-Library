\chapter{Acl.Crypto.Symmetric.Algorithm.ECMQV}

Bei diesem generischen Paket handelt es sich um eine Implementierung des
ECMQV-Schl�sselgenerierungsalgorithmus. Der ECMQV-Algorithmus ist die Erweiterung des MQV-Algorithmus 
auf den Bereich der Elliptischen Kurven.\\
Mit Hilfe von ECMQV sind zwei Parteien in der Lage ein gemeinsamen geheimen Schl�ssel 
aus den eigenen Schl�sselpaar und aus dem �ffentlichen Schl�ssel der anderen Partei zu erzeugen.\\
Dieser Algorithmus ist patentrechtlich gesch�tzt zur Verwendung muss die Genehmigung
Eigent�mers eingeholt werden.

\section{Mathematische Beschreibung}
\subsection{Voraussetzung}
Eine Elliptische Kurve $E$ der Form $y^2 = x^3 + a*x + b mod p$. \\
Ein Punkt $P$ der auf der Kurve $E$ liegt und  von der Ordnung $n$ ist.
\\~\\
Ein Schl�sselpaar besteht aus einem �ffentlichen Punkt $Q$ und einer geheimen Zahl $d$ welche die Gleichung $Q = d * P$ erf�llen.
\\~\\
Es werden also folgende Parameter ben�tigt: $a$, $b$, $p$, $P$ und $n$. Sowie zwei Schl�sselpaare $(d_1,Q_1)$ und $(d_2,Q_2)$.
\\~\\
Die �ffentlichen Schl�ssel der anderen Partei werden mit $'$ gekennzeichnet.

\subsection{Schl�sselgenerierung}
\begin{enumerate}
  \item Berechne $h = 2^{ (log_2  r) / 2}$

	\item Wandle $x_{Q2}$ in eine Zahl $i$ um. 
  \item Berechne $t = i mod h$.
  \item Berechne $t = t + h$.

	\item Wandle $x_{Q2}'$ in eine Zahl $i'$ um. 
  \item Berechne $t' = i' mod h$.
  \item Berechne $t' = t' + h$.

	\item Berechne $e = t * d_1 + d_2 mod n$.
	
	\item Berechne einen Punkt $G$ mit $G = e(Q_2' + t' * Q_1')$. $G$ darf nicht der Punkt im Unendlichen sein.
	\item Wandle $x_{G}$ in eine Zahl $z$ um. 
	\item Gib $z$ als gemeinsamen Schl�ssel aus.
\end{enumerate}

\section{Generischer Teil}
\begin{lstlisting}{}
generic
  Size : Positive;
\end{lstlisting}\ \\
\textbf{Exception}
$Size \not= 512+i64 \quad (i \in \{0, \ldots , 8\})$ : Constraint\_Error;\\

\section{API}
\subsection{Typen}
\begin{lstlisting}{}
 type Public_Key_ECMQV;
 type Private_Key_ECMQV is private;
 type Shared_Key_ECMQV;
\end{lstlisting}
Der \textit{Public\_Key\_ECMQV}  ist ein beinhaltet folgende Variablen:
\begin{lstlisting}{}
 E : Elliptic_Curve_Zp;
 P : EC_Point;
 n : Big_Unsigned;
 Q : EC_Point;
\end{lstlisting}
\textit{E} ist eine elliptische Kurve �pber ZP. \textit{P} ist ein �ffentlicher Punkt auf dieser Kurve, dessen Ordnung \textit{n} bekannt ist.
\textit{Q} ist der �ffenliche Teil des Schl�ssels. 
\\~\\
Der \textit{Private\_Key\_ECMQV} beinhaltet folgende Variablen:
\begin{lstlisting}{}
 Q : EC_Point;
 d : Big_Unsigned;
\end{lstlisting}
\textit{Q} ist ein Punkt auf der Elliptischen Kurve und der �ffenliche Teil des Schl�ssels. \textit{n} ist eine Zahl und der geheime Teil des Schl�ssels. 
\\~\\
Die \textit{Shared\_Key\_ECMQV} enth�lt einen Punkt. 
\begin{lstlisting}{}
 z : Big_Unsigned;
\end{lstlisting}



%%%%%%%%%%%%%%%%%%%%%%%%%%%%%%%%%%%%%%%%%%%%%%%%%%%%%%%%%%%%%%%%%%%%%%%%%%%
%%%%%%%%%%%%%%%%%%%%%%%%%%%%%%%%%%%%%%%%%%%%%%%%%%%%%%%%%%%%%%%%%%%%%%%%%%%


\subsection{Prozeduren und Funktionen}

\begin{tabular}{p{\textwidth}}
\begin{lstlisting}{}
procedure Gen_Single_Public_Key(Public_Key_A : out Public_Key_ECMQV;
                                length       : in DB.Bit_Length);
\end{lstlisting}\\
Diese Prozedur holt eine Elliptische Kurve aus der Datenbank. Die Kurve hat mindestens die kryptografische Sicherheit von \textit{length}.\\ \ \\
\textbf{Exception:}\\
\begin{tabular}{l @{\ :\ } l}
  BitLength is not supported. (Max BitLength = 521) & LEN\_EX\\
\end{tabular}\ \\ \ \\
\hline
\end{tabular}

%%%%%%%%%%%%%%%%%%%%%%%%%%%%%%%%%%%%%%%%%%%%%%%%%%%%%%%%%%%%%%%%%%%%%%%%%%%

\begin{tabular}{p{\textwidth}}
\begin{lstlisting}{}
procedure Gen_Public_Keys(Public_Key_A_1 : out Public_Key_ECMQV;
                          Public_Key_A_2 : out Public_Key_ECMQV;
                          length         : in DB.Bit_Length);
\end{lstlisting}\\
Diese Prozedur holt zwei Elliptische Kurve aus der Datenbank. Eine Kurve hat mindestens die kryptografische Sicherheit von \textit{length}. 
Hierzu wird die Funktion \textit{Gen\_Single\_Public\_Key} genutzt.
\\ \ \\
\textbf{Exception:}\\
\begin{tabular}{l @{\ :\ } l}
  BitLength is not supported. (Max BitLength = 521) & LEN\_EX\\
\end{tabular}\ \\ \ \\
\hline
\end{tabular}

%%%%%%%%%%%%%%%%%%%%%%%%%%%%%%%%%%%%%%%%%%%%%%%%%%%%%%%%%%%%%%%%%%%%%%%%%%%

\begin{tabular}{p{\textwidth}}
\begin{lstlisting}{}
procedure Gen_Single_Private_Key(Public_Key_A  : in out Public_Key_ECMQV;
                                 Private_Key_A : out    Private_Key_ECMQV);
\end{lstlisting}\\
Diese Prozedur erzeugt ein Schl�sselpaar, den \textit{Private\_Key\_A}. Hierzu wird eine zuf�llige Zahl generiert, diese ist der geheime Teil des Schl�ssels. 
Des weiteren wird ein Punkt als �ffentlicher Teil des Schl�ssels, als Produkt aus dem eben generierten Zahl und dem �ffentlichen Punkt des \textit{Public\_Key\_A} berechnet. Dieser Punkt wird sowohl im �ffentlichen als auch im privaten Schl�ssel gespeichert.
\\ \ \\
\hline
\end{tabular}

%%%%%%%%%%%%%%%%%%%%%%%%%%%%%%%%%%%%%%%%%%%%%%%%%%%%%%%%%%%%%%%%%%%%%%%%%%%

\begin{tabular}{p{\textwidth}}
\begin{lstlisting}{}
procedure Gen_Private_Keys(Public_Key_A_1  : in out Public_Key_ECMQV;
                           Public_Key_A_2  : in out Public_Key_ECMQV
                           Private_Key_A_1 : out    Private_Key_ECMQV;
                           Private_Key_A_2 : out    Private_Key_ECMQV);
\end{lstlisting}\\
Diese Prozedur erzeugt zwei Schl�sselpaare \textit{Private\_Key\_A\_1} und \textit{Private\_Key\_A\_2}. Hierzu wird die Funktion \textit{Gen\_Single\_Private\_Key} genutzt.
\\ \ \\
\hline
\end{tabular}

%%%%%%%%%%%%%%%%%%%%%%%%%%%%%%%%%%%%%%%%%%%%%%%%%%%%%%%%%%%%%%%%%%%%%%%%%%%

\begin{tabular}{p{\textwidth}}
\begin{lstlisting}{}
procedure Gen_Shared_Private_Key(Public_Key_B_1  : in  Public_Key_ECMQV;
                                 Public_Key_B_2  : in  Public_Key_ECMQV;
                                 Private_Key_A_1 : in  Private_Key_ECMQV;
                                 Private_Key_A_2 : in  Private_Key_ECMQV;
                                 Shared_Key_A    : out Shared_Key_ECMQV);
\end{lstlisting}\\
Diese Prozedur erzeugt einen �ffentlichen Schl�ssel, den \textit{Shared\_Key\_A} aus den Parametern.
\\ \ \\
\hline
\end{tabular}

%%%%%%%%%%%%%%%%%%%%%%%%%%%%%%%%%%%%%%%%%%%%%%%%%%%%%%%%%%%%%%%%%%%%%%%%%%%

\begin{tabular}{p{\textwidth}}
\begin{lstlisting}{}
function Verify(Public_Key_A_1  : Public_Key_ECMQV;
                Public_Key_A_2  : Public_Key_ECMQV;
                Public_Key_B_1  : Public_Key_ECMQV;
                Public_Key_B_2  : Public_Key_ECMQV;
                Private_Key_A_1 : Private_Key_ECMQV;
                Private_Key_A_2 : Private_Key_ECMQV;
                Private_Key_B_1 : Private_Key_ECMQV;
                Private_Key_B_2 : Private_Key_ECMQV;
                Shared_Key_A    : Shared_Key_ECMQV;
                Shared_Key_B    : Shared_Key_ECMQV) return Boolean;
\end{lstlisting}\\
Diese Funktion gibt den Wert ``True'' zur�ck, wenn mit Hilfe sich auf Basis aller Informationen
der der gleich geheime Schl�ssel f�r beide Parteien berechnet werden konnte.
\\ \ \\
\hline
\end{tabular}


%%%%%%%%%%%%%%%%%%%%%%%%%%%%%%%%%%%%%%%%%%%%%%%%%%%%%%%%%%%%%%%%%%%%%%%%%%%

\section{Anwendungsbeispiel}
\begin{lstlisting}{}
with Crypto.Types.Big_Numbers;
with Crypto.Asymmetric.ECMQV;
with Ada.Text_IO; use Ada.Text_IO;

procedure Example_ECMQV is
   package ECMQV is new Crypto.Asymmetric.ECMQV(512);
   use ECMQV;

   Public_Key_A_1  : Public_Key_ECMQV;
   Public_Key_A_2  : Public_Key_ECMQV;
   Public_Key_B_1  : Public_Key_ECMQV;
   Public_Key_B_2  : Public_Key_ECMQV;
   Private_Key_A_1 : Private_Key_ECMQV;
   Private_Key_A_2 : Private_Key_ECMQV;
   Private_Key_B_1 : Private_Key_ECMQV;
   Private_Key_B_2 : Private_Key_ECMQV;
   Shared_Key_A    : Signature_ECMQV;
   Shared_Key_B    : Signature_ECMQV;

begin
   --Schluesselgenerierung
   Gen_Public_Keys(Public_Key_A_1, Public_Key_A_2, 178);
   Gen_Public_Keys(Public_Key_B_1, Public_Key_B_2, 178);

   Gen_Private_Keys(Public_Key_A_1, Public_Key_A_2, 
                    Private_Key_A_1, Private_Key_A_2);
   Gen_Private_Keys(Public_Key_B_1, Public_Key_B_2, 
                    Private_Key_B_1, Private_Key_B_2);

   --ECMQV-Schluesselgenerierung
   Gen_Shared_Private_Key(Public_Key_B_1, Public_Key_B_2, 
                          Private_Key_A_1, Private_Key_A_2, 
                          Shared_Key_A);
   Gen_Shared_Private_Key(Public_Key_A_1, Public_Key_A_2, 
                          Private_Key_B_1, Private_Key_B_2, 
                          Shared_Key_B);

   --Verifikation
   if Verify(Public_Key_A_1, Public_Key_A_2, 
             Public_Key_B_1, Public_Key_B_2, 
             Private_Key_A_1, Private_Key_A_2, 
             Private_Key_B_1, Private_Key_B_2, 
             Shared_Key_A,Shared_Key_B) then
        Put_Line("OK");
   else Put_Line("Implementation error.");
   end if;

end Example_ECMQV;
\end{lstlisting}
