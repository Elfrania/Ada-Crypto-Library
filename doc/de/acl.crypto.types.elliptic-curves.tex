\chapter{Ellyptische Kurven}
Die ACL unterst�tzt die beiden folgenden Klasen von ellyptischen Kurven
\begin{itemize}
\item $y^2 = x^3 + ax + b\; \bmod\; p \mbox{ mit } p \in \mathbf{P}
  \setminus\{2,3\}$
  \\
  Hier handelt es sich um ellyptische Kurven �ber dem endlichen K�rper $Z_p$.
\item  $y^2 + xy = x^3 + a(z)x^2 + b(z) \;\bmod\; f(z)$\\
  Hier handelt es sich �ber nicht supersingulare ellyptische Kurven �ber 
  den endlichen K�rper $GF(2^{\deg{f(z)}})$.\\
  \textbf{Vorraussetzung:} $f(z)$ ein irreduzibles Polynom.\\
\end{itemize}

\section{Wurzelpaket: Types.Elliptic\_Curves}
Dieses generische Paket ist das Wurzelpaket f�r ellyptische Kurven.
In ihm ist der Basistype \texttt{EC\_Point} f�r ellyptische Kurven definiert.

\subsection{API}

\subsubsection{Generischer Teil}
\begin{lstlisting}{}
  generic
  with package Big is new Crypto.Types.Big_Numbers(<>);
\end{lstlisting}

\subsubsection{Typen}
Dieses Paket stellt den Typ \textit{EC\_Point} zur Verf�gung.
\begin{lstlisting}{}
  -- (x,y)
  type EC_Point is record
     X : Big.Big_Unsigned;
     Y : Big.Big_Unsigned;
  end record;
\end{lstlisting}\ \\
Die Konstante \texttt{EC\_Point\_Infinity} entspricht dem Punkt $\infty$.
Dieser Punkt befindet sich auf jeder ellyptischen Kurve und entspricht dem
neutrale Element der Addition.


\subsubsection{Prozeduren}
\begin{lstlisting}{}
  procedure Put(Item : in EC_Point; Base : in Big.Number_Base := 10;
  
  procedure Put_Line(Item : in EC_Point; Base : in Big.Number_Base := 10);
\end{lstlisting}
Die beiden Prozeduren geben einen \texttt{EC\_Point} auf der Standardausgabe
in Form eines Tupels aus.\\ 
\ \\
\textbf{Beispiel:}
\begin{lstlisting}{}
  procedure Example is
    P : EC_Point(X => Big_unsigned_Three, Y => Big_Unsigned_one)
  begin
    Put(P);             -- Ausgabe: "(3,1)"
    Put(P, Base => 2);  -- Ausgabe: "(2#11, 2#1#)"
  end Example;
\end{lstlisting}\ \\ \ \\

\section{Kinderpakete}
Die beiden Kinderpakete
\begin{itemize}
  \item Types.Elliptic\_Curves.Zp
  \item Types.Elliptic\_Curves.NSS\_BF
\end{itemize}
verf�gen �ber die folgende Typen

\subsubsection{Typen}
Das Paket \textit{Types.Elliptic\_Curves.Zp} stellt den Typ \textit{Elliptic\_Curve\_Zp} zur Verf�gung.
\begin{lstlisting}{}
  -- (A,B,P)
  type Elliptic_Curve_Zp is record
    A : Big_Unsigned;
    B : Big_Unsigned;
    P : Big_Unsigned;
  end record;
\end{lstlisting}\ \\


\subsubsection{API}

\begin{tabular}{p{\textwidth}}
\begin{lstlisting}{}
   procedure Init(A, B, P : in Big_Unsigned);
   procedure Init(ECZ : in Elliptic_Curve_Zp);

   procedure Init(A, B, F : in Big_Unsigned);
\end{lstlisting}\\
Diese Prozeduren initialiseren die ellyptische Kurven.
Die ersten beiden bilden die ellyptische Kurve 
$y^2 = x^3 + Ax + B\; \bmod\; P$, die dritte die Kurve
$y^2 + xy = x^3 + Ax^2 + B \;\bmod\; F.$\\
Diese Funktion testet nicht, ob es sich bei \texttt{P} bzw. \texttt{F}
wirklich um eine Primzahl bzw. irreduzibles Polynom handelt oder die
Parameter tats�chlich eine ellyptische Kurve bilden.\\ \ \\
\hline
\end{tabular}   

%%%%%%%%%%%%%%%%%%%%%%%%%%%%%%%%%%%%%%%%%%%%%%%%%%%%%%%%%%%%%%%%%%%%%%%%%%%

\begin{tabular}{p{\textwidth}}
  \begin{lstlisting}{}
    function Is_Elliptic_Curve return Boolean;
  \end{lstlisting}\\
  Diese Funktion testet, ob durch die Initalisierung mittels \texttt{Init}
  ein ellyptische Kurve konstruiert worden ist. Wenn dies nicht der Fall ist,
  dann liefern alle folgenden Funktionen und Prozeduren wahrscheinlich 
  nicht das gew�nschte Ergebnis.\\ \ \\
  \hline
\end{tabular}   

%%%%%%%%%%%%%%%%%%%%%%%%%%%%%%%%%%%%%%%%%%%%%%%%%%%%%%%%%%%%%%%%%%%%%%%%%%%

\begin{tabular}{p{\textwidth}}
  \begin{lstlisting}{}
    function On_Elliptic_Curve(X : EC_Point) return Boolean;
  \end{lstlisting}
  Diese Funktion testet, ob \texttt{X} ein Punkt auf der eyllptischen Kurve 
  ist.\\ \ \\
\end{tabular}

%%%%%%%%%%%%%%%%%%%%%%%%%%%%%%%%%%%%%%%%%%%%%%%%%%%%%%%%%%%%%%%%%%%%%%%%%%%

\begin{tabular}{p{\textwidth}}
  \begin{lstlisting}{}
    function Negative(X : EC_Point) return EC_Point;
  \end{lstlisting}
  Dies Funktion berechnet \texttt{-X}.\\ \ \\
\end{tabular}

%%%%%%%%%%%%%%%%%%%%%%%%%%%%%%%%%%%%%%%%%%%%%%%%%%%%%%%%%%%%%%%%%%%%%%%%%%%

\begin{tabular}{p{\textwidth}}
  \begin{lstlisting}{}
    function "+"(Left, Right : EC_Point) return EC_Point;
  \end{lstlisting}
  Dies Funktion berechnet \texttt{Left + Right}.\\ \ \\
  \hline
\end{tabular}

%%%%%%%%%%%%%%%%%%%%%%%%%%%%%%%%%%%%%%%%%%%%%%%%%%%%%%%%%%%%%%%%%%%%%%%%%%%

   \begin{tabular}{p{\textwidth}}
  \begin{lstlisting}{}
    function Double(X : EC_Point) return EC_Point;
  \end{lstlisting}
  Dies Funktion berechnet \texttt{2X}.\\ \ \\
  \hline
   \end{tabular}


