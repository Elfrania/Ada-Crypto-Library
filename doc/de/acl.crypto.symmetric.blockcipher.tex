\chapter{Crypto.Symmetric.Blockcipher}\label{block}
Mit Hilfe dieses generischen Paketes l�sst sich aus dem Algorithmus einer
symmetrischen Chiffre (siehe Kapitel \ref{algobc}) eine Blockchiffre erstellen.
Sie sollten davon Abstand nehmen die API dieses Paketes direkt zu 
verwenden, denn dieses Paket implementiert eine Blockchiffre im ``unsicheren''
ECB-Modus (Electronic Codebook Modus). 
Wenn man in diesem Modus zwei identische Klartextbl�cke  p1=p2 mit dem selben
Schl�ssel chiffriert, dann erh�lt man zwei identische Chiffretextbl�cke  c1=c2.
Dadurch kann der Chiffretext noch Informationen �ber die Struktur des 
Klartextes enthalten. 

\section{API}

Die API eine Blockchiffre besteht aus drei Prozeduren.

\begin{enumerate}
\item Die Prozedur \textbf{Prepare\_Key} weist einer Blockchiffre
  einen Schl�ssel \textit{Key} zu.
  \begin{lstlisting}{}
 procedure Prepare_Key(Key : in Key_Type);
  \end{lstlisting}
  
\item Die Prozedur \textbf{Encrypt} verschl�sselt (mit Hilfe eines vorher 
  zugewiesenen Schl�ssels) einen Klartextblock  (Plaintext) in einen 
  Chiffretextblock (Ciphertext) 
  \begin{lstlisting}{}
 procedure Encrypt(Plaintext : in Block; Ciphertext : out Block);
  \end{lstlisting}

\item Die Prozedur \textbf{Decrypt} entschl�sselt (mit Hilfe eines vorher
  zugewiesenen Schl�ssels) einen Chiffretextblock (Ciphertext) in einen 
  Klartextblock (Plaintext).
  \begin{lstlisting}{}
 procedure Decrypt(Ciphertext : in Block; Plaintext  : out Block);
  \end{lstlisting}

\end{enumerate}

\section{Generischer Teil}

 \begin{lstlisting}{}
 type Block is private;
 type Key_Type is private;
 type Cipherkey_Type is private;

 with procedure Prepare_Key(Key       : in  Key_Type;
                            Cipherkey : out Cipherkey_Type);

 with procedure Encrypt(Cipherkey  : in  Cipherkey_Type;
                        Plaintext  : in  Block;
                        Ciphertext : out Block);

 with procedure Decrypt(Cipherkey  : in  Cipherkey_Type;
                        Ciphertext : in  Block;
                        Plaintext  : out Block);
 \end{lstlisting}

\section{Anwendungsbeispiel}
\begin{lstlisting}{generic TDES}
with Crypto.Types;
with Crypto.Symmetric.Blockcipher;
with Crypto.Symmetric.Algorithm.TripleDES;

procedure Generic_TripleDES is
use Crypto.Types;
use Crypto.Symmetric.Algorithm.TripleDES;

   package Generic_TripleDES is 
      new  Crypto.Symmetric.Blockcipher
      (Block          => B_Block64,
       Key_Type       => B_Block192,
       Cipherkey_Type => Cipherkey_TDES,
       Prepare_Key    => Prepare_Key,
       Encrypt        => Encrypt,
       Decrypt        => Decrypt);
begin
   ...
end Generic_TripleDES;
\end{lstlisting}

\section{Anmerkung}
Sie m�ssen nicht jedes mal von neuem aus einem symmetrischen Algorithmus eine
Blockchiffre generiern. Statdessen k�nnen Sie auch eine der folgenden 
vorgefertigten  Blockchiffren verwenden.
\begin{itemize}
\item Crypto.Symmetric.Blockcipher\_AES128
\item Crypto.Symmetric.Blockcipher\_AES192
\item Crypto.Symmetric.Blockcipher\_AES256
\item Crypto.Symmetric.Blockcipher\_Serpent256
\item Crypto.Symmetric.Blockcipher\_Tripledes
\item Crypto.Symmetric.Blockcipher\_Twofish128
\item Crypto.Symmetric.Blockcipher\_Twofish192
\item Crypto.Symmetric.Blockcipher\_Twofish256
\end{itemize}



