\chapter{Crypto.Asymmetric.ECDH}\label{ECDH}
In this generic package the implementation of Elliptic Curve Diffie–Hellman (ECDH) algorithm is provided. The ECDH is an key agreement protocol that allows two parties, each having the other party's public key, and one's own private key, to establish a shared secret over an insecure channel \cite{ECDH-NIST}. A brief description of elliptic curve will be introduced here. For detailed information you can go to \texttt{Crypto.Types.Elliptic\_Curve} (Chapter \ref{ConceptofElliptic}).
\subsubsection*{Mathematical Background}
An elliptic curve $E$ is of the form $y^2=x^3+ax+b\mod p$. A point $P$ is in the curve $E$ and has an order $n$. A pair of keys are made of a public point $Q$ and a secret number $d$ where $Q=d*P$. For calculation these following parameters are required: $a,b,p,P,n$ and also the key pair $(d,Q)$.
\subsubsection*{Key Generation}\\
Let Alice's key pair be ($d_A,Q_A$) and Bob's key pair be ($d_B,Q_B$). Each public part should be known to the other party.\\
Alice compute $(x_k,y_k)=d_AQ_B$, Bob computes $(x_k,y_k)=d_BQ_A$. The shared key is $x_k$ (the $x$ coordinate of the point). The shared key calculated by both parties is equal, because\\ $d_AQ_B=d_Ad_BP=d_Bd_AP=d_BQ_A$.
\section{API}
\subsubsection*{Generic Part}
\begin{lstlisting}{}
  generic
    Size : Positive;
\end{lstlisting}
\textbf{Types}
\begin{lstlisting}{}
  type Private_Key_ECDH is private;
  type Shared_Key_ECDH;
  type Public_Key_ECDH;
\end{lstlisting}
The type \texttt{Private\_Key\_ECDH} has two components:
\begin{lstlisting}{}
  type ECDH_S_KEY is record
    Q : EC_Point; 
    d : Big_Unsigned;
  end record;
  type Private_Key_ECDH is new ECDH_S_KEY;
\end{lstlisting}
$Q$ is a point in the elliptic curve and also the public part of the key. $d$ is a number and the secret part of the key.\\
The type \texttt{Shared\_Key\_ECDH} contains a point:
\begin{lstlisting}{}
  type ECDH_KEY is record
    W : EC_Point;  --(x,y) in Key Generation
  end record;
  type Shared_Key_ECDH is new ECDH_KEY;
\end{lstlisting}\\
The type \texttt{Public\_Key\_ECDH} is made of:
\begin{lstlisting}{}
  type ECDH_P_KEY is record
    E : Elliptic_Curve_Zp;
    P : EC_Point;
    n : Big_Unsigned;
    Q : EC_Point;
  end record;
  type Public_Key_ECDH is new ECDH_P_KEY;
\end{lstlisting}\\
$E$ is an elliptic curve over $Z_P$. $P$ is a public point in the curve and is in order $n$, $Q$ is the public part of the key.\\
\subsubsection*{Procedures}
\begin{lstlisting}{}
  procedure Gen_Public_Key(Public_Key_A  : out Public_Key_ECDH;
			    			   	length        : in  DB.Bit_Length);
\end{lstlisting}
This procedure gets an elliptic curve from the database. The curve has at least cryptographic security in length.\\
%\textbf{Exception:}
%If length in bit is not supported (max. bit length = 512): \texttt{LEN\_EX}.\\
\hline \\ \ \\
\begin{lstlisting}{}
  procedure Gen_Single_Private_Key
  					 (Public_Key_A  : in out Public_Key_ECDH;
				     Private_Key_A : out    Private_Key_ECDH);
\end{lstlisting}
In this procedure, a private key \texttt{Private\_Key\_A} will be created according to the value \texttt{Public\_Key\_A}. Additionally a random number $d$ is generated as the secret part of \texttt{Private\_\-Key\_A}. Then a point $Q$ is calculated as the public part of \texttt{Private\_Key\_A} which is the product of the random number and the public point $P$ of \texttt{Public\_Key\_A} (that is, $Q=d*P$). Point $Q$ will be stored in both keys.\\
\hline \\ \ \\
\begin{lstlisting}{}
  procedure Gen_Shared_Private_Key
  					(Public_Key_B  : in Public_Key_ECDH;
				    Private_Key_A : in Private_Key_ECDH;
				    Shared_Key_A  : out Shared_Key_ECDH);
\end{lstlisting}
In this procedure a shared key \texttt{Shared\_Key\_A} is generated. It is calculated as the product of the public part of \texttt{Public\_Key\_B} and the secret part of \texttt{Private\_Key\_A}, ($Q_B*d_A$).\\
\textbf{Exception:}\\
If the shared key is equal infinity:\quad \texttt{EC\_IFINITY\_EX}.\\
\hline \\ \ \\
\begin{lstlisting}{}
  function Verify(Public_Key_A  : Public_Key_ECDH;
		   			Public_Key_B  : Public_Key_ECDH;
		   			Private_Key_A : Private_Key_ECDH;
		   			Private_Key_B : Private_Key_ECDH;
		   			Shared_Key_A  : Shared_Key_ECDH;
		   			Shared_Key_B  : Shared_Key_ECDH) 
		   			                return Boolean;
\end{lstlisting}
This function returns true if the same private keys and shared keys can be calculated from the delivered informations.\\
\section{Example}
\begin{lstlisting}{}
  with Crypto.Types.Big_Numbers;
  with Ada.Text_IO;
  with Crypto.Symmetric.Algorithm.ECDH;
  use Ada.Text_IO;
  procedure Example_ECDH is
    package ECDH is new Crypto.Symmetric.Algorithm.ECDH(512);
    use ECDH;
    Public_Key_A, Public_Key_B : Public_Key_ECDH;
    Private_Key_A, Private_Key_B : Private_Key_ECDH;
    Shared_Key_A, Shared_Key_B : Shared_Key_ECDH;
  begin
     -- Key Generation
    Gen_Public_Key(Public_Key_A, 178);
    Gen_Single_Private_Key(Public_Key_A, Private_Key_A);
    Gen_Public_Key(Public_Key_B, 178);
    Gen_Single_Private_Key(Public_Key_B, Private_Key_B);

    Gen_Shared_Private_Key(Public_Key_B,Private_Key_A,Shared_Key_A);
    Gen_Shared_Private_Key(Public_Key_A,Private_Key_B,Shared_Key_B);
    
    if Verify(Public_Key_A, Public_Key_B, Private_Key_A, 
              Private_Key_B, Shared_Key_A, Shared_Key_B) then
       Put_Line("OK");
    else 
       Put_Line("Error");
    end if;
  end Example_ECDH;
\end{lstlisting}
%%%%%%%%%%%%%%%%%%%%%%%%%%%%%%%%%%%%%%%%%%%%%%%%%%%%%%%%%%%%%%%%%%%%%%%
%%%%%%%%%%%%%%%%%%%%%%%%%%%%%%%%%%%%%%%%%%%%%%%%%%%%%%%%%%%%%%%%%%%%%%%
\chapter{Crypto.Asymmetric.ECIES}
Elliptic Curve Integrated Encryption Scheme (ECIES) is proposed by Abdalla, Bellare and Rogaway in \cite{DBLP:conf/ctrsa/AbdallaBR01}, and is one of the two standardized IES which based on elliptic curves. Using ECIES algorithm two parties can do encryption and decryption with a shared private key. Usually ECDH (Chapter \ref{ECDH}) algorithm will be used to generate the shared key.
\subsubsection*{Precondition:}
The following algorithms will be used to calculate a shared key:
\begin{itemize}
\item Hash algorithm\quad\quad : SHA-256
\item Encryption algorithm : AES-256
\item MAC algorithm\quad\quad : HMAC-SHA-256
\end{itemize}
\subsubsection*{Encryption}
\begin{enumerate}
\item Two keys are generated from the shared ECDH key, one ($k_E$) for encryption and the other one ($k_M$) for MAC algorithm.
\item The message $M$ is encrypted: $C=E({k_E},M)$.
\item Tag of the ciphertext is calculated: $T=MAC({k_M},C)$.
\item Output: $C||T$.
\end{enumerate}
\subsubsection*{Decryption}
\begin{enumerate}
\item Two keys ($k_E$ and $k_M$) will be generated again. 
\item Use MAC to check the tag if $T'=MAC({k_M},C)$.
\item Decrypt the message: $M=D({k_E},C)$.
\end{enumerate}
\section{API}
\subsubsection*{Generic Part}
\begin{lstlisting}{}
  generic
    Size : Positive;
\end{lstlisting}
\textbf{Exception}: $Size\neq 512+64*I$ $(I\in \{0,\cdots,8\})$: \texttt{Constraint\_Error}.\\
\subsubsection*{Types}
\begin{lstlisting}{}
  type Cipher_ECIES is private;
  type Plain_ECIES is private;
\end{lstlisting}
The type \texttt{Cipher\_ECIES} has five components:
\begin{lstlisting}{}
  type ECIES_C_KEY is record
    Public_Point        : EC_Point;
    Message_Block_Count : natural := 0;
    Mac                 : W_Block256;
    Cipher              : Unbounded_String;
    Cipher_Map          : Container_Cipher_Map.Map;
  end record;
  type Cipher_ECIES is new ECIES_C_KEY;
\end{lstlisting}
The term \texttt{Public\_Point} is the public key. \texttt{Message\_Block\_Count} is used to count the ciphertext blocks. \texttt{Mac} is the MAC tag of the ciphertext block. \texttt{Cipher} is the ciphertext. The ciphertext is divided into blocks, the blocks are indexed in \texttt{Cipher\_Map}.\\ \ \\
The type \texttt{Plain\_ECIES} is made of:
\begin{lstlisting}{}
  type ECIES_P_KEY is record
    Message_Block_Count : natural := 1;
    AES_Key             : B_Block256;
    Mac_Key             : W_Block512;
    Message             : Unbounded_String;
    Message_Map         : Container_Message_Map.Map;
  end record;
  type Plain_ECIES is new ECIES_P_KEY;
\end{lstlisting}
\texttt{Message\_Block\_Count} is the count of the message blocks. \texttt{AES\_Key} is the key for AES-256 algorithm, and \texttt{Mac\_Key} is the key for HMAC-SHA-256 algorithm. \texttt{Message} is the plaintext. The plaintext is divided into blocks, and they are indexed in \texttt{Message\_Map}.\\
\subsubsection*{Procedures}
\begin{lstlisting}{}
   --internal purpose
  procedure Message_Prepare(Plain   : out Plain_ECIES;
                            Message : in  String);
\end{lstlisting}
This procedure converts a string to the type \texttt{Plain\_ECIES}. It's only for internal use.\\
\hline \\ \ \\
\begin{lstlisting}{}
   --internal purpose
  procedure Key_Prepare(AES_Key    : out B_Block256;
                        Mac_Key    : out W_Block512;
                        Shared_Key : in Shared_Key_ECDH);
\end{lstlisting}
This procedure generates a key \texttt{AS\_Key} for AES-256 algorithm and a key \texttt{Mac\_Key} for HMAC-SHA-256 algorithm based on the \texttt{Shared\_Key}. It's only for internal use.\\
\hline \\ \ \\
\begin{lstlisting}{}
   --internal purpose
  procedure Mac_Compute(Mac_Key : in  W_Block512;
                        Cipher  : in  Cipher_ECIES;
                        Mac     : out W_Block256);
\end{lstlisting}
This procedure computes a tag of the ciphertext. HMAC-SHA-256 algorithm and the \texttt{Mac\_Key} are used. It's only for internal use.\\
\hline \\ \ \\
\begin{lstlisting}{}
  procedure Encrypt(Public_Key_A : in  Public_Key_ECDH;
                    Shared_Key   : in  Shared_Key_ECDH;
                    Plaintext	   : in  String;
                    Cipher       : out Cipher_ECIES);
  procedure Decrypt(Shared_Key	: in  Shared_Key_ECDH;
                    Cipher       : in  Cipher_ECIES;
                    Plaintext    : out Unbounded_String);

\end{lstlisting}
In the procedure \texttt{Encrypt()} all the internal helping procedures are called. The \texttt{AES\_Key} and \texttt{Mac\_Key} are generated from the \texttt{Shared\_Key}. Then the message is encrypted  and a tag will be calculated at last. 
Procedure \texttt{Decrypt()} does much the same as in \texttt{Encrypt()}, additionally it makes a verification of the generated tag and the tag of the \texttt{Cipher}, if the two tags are equal, then the plaintext is returned, else an exception will be raised with "Found different MACs".\\
\textbf{Exception:}\\ 
If a different MAC is found:\quad \texttt{MAC\_ERROR};\\
\hline \\ \ \\
\begin{lstlisting}{}
  procedure Encrypt(Public_Key_A  : in  Public_Key_ECDH;
                    Public_Key_B  : in  Public_Key_ECDH;
                    Private_Key_A : in  Private_Key_ECDH;
                    Plaintext	    : in  String;
                    Cipher        : out Cipher_ECIES);
  procedure Decrypt(Public_Key_B  : in  Public_Key_ECDH;
                    Private_Key_A : in  Private_Key_ECDH;
                    Cipher        : in  Cipher_ECIES;
                    Plaintext     : out Unbounded_String);
\end{lstlisting}
In the procedure \texttt{Encrypt()} a shared key will be generated from \texttt{Public\_Key\_B} and \texttt{Private\-\_Key\_A}, and then it calls the procedure \texttt{Encrypt()} above to make encryption. Furthermore it tests the two variables of type \texttt{Public\_Key\_ECDH} for equality.\\
Procedure \texttt{Decrypt()} decrypts the ciphertext by generating the shared key from \texttt{Public\_K\-ey\_B()} and \texttt{Private\_Key\_A()} and then calling the procedure \texttt{Decrypt()} above.\\
\textbf{Exception:}\\ 
If a different elliptic curve or domain parameter is found:\quad \texttt{CURVE\_ERROR}.\\
\section{Example}
\begin{lstlisting}{}
  with Crypto.Symmetric.Algorithm.ECIES;
  with Ada.Text_IO; use Ada.Text_IO;
  with Ada.Strings.Unbounded;
  use Ada.Strings.Unbounded;
  procedure Example_ECIES is
    package ECIES is new Crypto.Symmetric.Algorithm.ECIES(512);
    use ECIES;
    Public_Key_A, Public_Key_B : ECIES.ECDH.Public_Key_ECDH;
    Private_Key_A, Private_Key_B : ECIES.ECDH.Private_Key_ECDH;
    Cipher : Cipher_ECIES;
    Input : String := "Hi there";
    Temp : Unbounded_String;
    Output : Unbounded_String;
    Result : Boolean := false;
  begin
    ECIES.ECDH.Gen_Public_Key(Public_Key_A, 178);
    ECIES.ECDH.Gen_Single_Private_Key(Public_Key_A, Private_Key_A);

    ECIES.ECDH.Gen_Public_Key(Public_Key_B, 178);
    ECIES.ECDH.Gen_Single_Private_Key(Public_Key_B, Private_Key_B);

    Encrypt(Public_Key_A, Public_Key_B, 
            Private_Key_A, Input, Cipher);
    Decrypt(Public_Key_B, Private_Key_A, Cipher, Output);

    Append(Temp, Input);
    for I in 1..Input'Length loop
       if To_String(Temp)(I) = Input(I) then
          Result := True;
       else
	       Result := False;
       end if;
    end loop;
    if Result then
       Put_Line("OK");
    else
       Put_Line("Error");
    end if;
  end Example_ECIES;
\end{lstlisting}
