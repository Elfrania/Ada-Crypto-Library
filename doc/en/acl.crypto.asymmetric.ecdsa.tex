\chapter{Crypto.Asymmetric.ECDSA}
In this generic package the ECDSA (Elliptic Curve DSA) algorithm is
implemented. It's a further extension of the DSA algorithm in the area
of elliptic curve. Users can generate digital signatures and verify
the authenticity of the signatures by using ECDSA. The private key is
used in the signature process, while the public key is used for
verification. During the signature process the SHA-1 hash value of the
message (not the message) is signed. People who hasn't the private key
can't generate a valid signature. However people who knows the public
key can convince the signature by running the verification process.
\section{API}
\subsubsection*{Generic Part}
\begin{lstlisting}{}
  generic
    Size : Positive;
\end{lstlisting}
\subsubsection*{Types}
\begin{lstlisting}{}
  type Public_Key_ECDSA is private;
  type Private_Key_ECDSA is private;
  type Signature_ECDSA is private;
\end{lstlisting}
The type \texttt{Public\_Key\_ECDSA} has four components:
\begin{lstlisting}{}
  type ECDSA_P_KEY is record
    E : Elliptic_Curve_Zp;
    P : EC_Point;
    n : Big_Unsigned;
    Q : EC_Point;
  end record;
  type Public_Key_ECDSA is new ECDSA_P_KEY;
\end{lstlisting}
$E$ is an elliptic curve over $Z_P$. $P$ is a public point in the
curve, and its order is known as $n$. $Q$ is the public part of the
key. \\ The type \texttt{Private\_Key\_ECDSA} has the following
variables:
\begin{lstlisting}{}
  type ECDSA_S_KEY is record
    Q : EC_Point;
    d : Big_Unsigned;
  end record;
  type Private_Key_ECDSA is new ECDSA_S_KEY;
\end{lstlisting}
$Q$ is a point in the elliptic curve and also the public part of the
key. $d$ is a number and the private part of the key.\\ The type
\texttt{Signature\_ECDSA} contains the following variables:
\begin{lstlisting}{}
  type ECDSA_KEY is record
    R : Big_Unsigned;
    S : Big_Unsigned;
  end record;
  type Signature_ECDSA is new ECDSA_KEY;
\end{lstlisting}
The signature is made of a pair ($R,S$).\\
\subsection*{Procedures}
\begin{lstlisting}{}
  procedure Gen_Public_Key(Public_Key  : out Public_Key_ECDSA;
			               	length      : in DB.Bit_Length);
\end{lstlisting}
This procedure gets an elliptic curve from the database. The curve has
at least cryptographic security in length.

\hhline
\begin{lstlisting}{}
 procedure Gen_Private_Key(Public_Key :in out Public_Key_ECDSA;
                           Private_Key:out Private_Key_ECDSA);
\end{lstlisting}
This procedure generates a private key corresponding to the previously
created public key. A random number ($d$) will be produced at first as
the secret part of the key. Then a point ($Q$) is calculated as common
part of the two keys (it is stored in both keys) from the random
number ($d$) and the public point of the public key ($P$).

\hhline
\begin{lstlisting}{}
  procedure Sign(Public_Key  : in Public_Key_ECDSA;
		 			  Private_Key : in  Private_Key_ECDSA;
                 SHA1_Hash   : in  W_Block160;
                 Signature   : out Signature_ECDSA);
\end{lstlisting}
This procedure signs the SHA-1 hash value of a message with the private key.\\

\hhline
\begin{lstlisting}{}
  function Verify(Public_Key  : Public_Key_ECDSA;
                  SHA1_Hash   : W_Block160;
                  Signature   : Signature_ECDSA)
                  return Boolean;
\end{lstlisting}
The function returns true if the signature can be verified with the
public key to be a valid signature of the SHA-1 hash value
\texttt{SHA1\_Hash}. If not, then false will be returned.\\ With the
public key only the signature which related to the private key can be
verified. If the two keys don't match together, then false will be
returned.

But consider this situation, if Bob uses the identity card of Alice,
then we can't make identification of Bob with the identity card,
although the identity card of Alice is valid.

\hhline
\begin{lstlisting}{}
  procedure Sign_File(Filename    : in  String;
          		       Public_Key  : in  Public_Key_ECDSA;
					     	 Private_Key : in  Private_Key_ECDSA;
						    Signature   : out Signature_ECDSA);
\end{lstlisting}
This procedure signs data file (\texttt{Filename}) with the private
key.\\

\noindent\textbf{Exception:}\\ If the \texttt{Private\_Key} is not
initialized:\quad \texttt{Invalid\_Private\_Key\_Error}.

\hhline
\begin{lstlisting}{}
  function Verify_File(Filename   : String;
					 		  Public_Key : Public_Key_ECDSA;
                       Signature  : Signature_ECDSA)
                       return Boolean;
\end{lstlisting}
This function returns true if the signature can be verified with the
public key to be a valid signature of the file (\texttt{Filename}). If
not, then false will be returned.\\ Only the signature which related
to the private key and the public key can be verified. If the two keys
don't match, then false will be returned.\\

\noindent\textbf{Exception:}\\ If the \texttt{Public\_Key} is not
initialized:\quad \texttt{Invalid\_Public\_Key\_Error}.\\


\hhline
\begin{lstlisting}{}
  function Verify_Key_Pair(Private_Key : Private_Key_ECDSA;
                           Public_Key  : Public_Key_ECDSA)
                           return Boolean;
\end{lstlisting}
This function returns true if the private key and the public key
match, that is, they are verified to be a pair of keys. If not, false
is returned.

\hhline
\begin{lstlisting}{}
  function equal_Public_Key_Curve
  					  					(Public_Key_A : Public_Key_ECDSA;
		              				 Public_Key_B : Public_Key_ECDSA)
								 			return Boolean;
\end{lstlisting}
This function returns true if the two public keys, \texttt{Public\_Key\_A} and \texttt{Public\_Key\_B}, are tested to be the same key, else false is returned.\\
\section{Example}
\begin{lstlisting}{}
  with Crypto.Asymmetric.ECDSA;
  with Crypto.Types.Big_Numbers;
  with Ada.Text_IO;
  use Ada.Text_IO;

  procedure Example_ECDSA is
  	package ECDSA is new Crypto.Asymmetric.ECDSA(512);
  	use ECDSA;
  	Public_Key : Public_Key_ECDSA;
  	Private_Key : Private_Key_ECDSA;
  	Signature : Signature_ECDSA;
  begin
    -- Generation
    Gen_Public_Key(Public_Key, 178);
  	 Gen_Private_Key(Public_Key, Private_Key);
  	-- Sign
  	 Sign_File("Example_DSA.adb", Public_Key,
  	          Private_Key, Signature);
  	-- Verification
  	if Verify_File("Example_DSA.adb", Public_Key, Signature) then
       Put_Line("OK");
    else
       Put_Line("Error");
    end if;
  end Example_ECDSA;
\end{lstlisting}
