\chapter{Crypto.Types}

This package provides the fundamental types of the ACL and their base 
functions.\\ 
\ \\
{\large \textbf{IMPORTANT:} \\
  When using the ACL you should in any case import this package via
  \begin{lstlisting}{}
    with Crypto.Types;
  \end{lstlisting}
  }

%%%%%%%%%%%%%%%%%%%%%%%%%%%%%%%%%%%%%%%%%%%%%%%%%%%%%%%%%%%%%%%%%%%%%%%%%%%
%%%%%%%%%%%%%%%%%%%%%%%%%%%%%%%%%%%%%%%%%%%%%%%%%%%%%%%%%%%%%%%%%%%%%%%%%%%
%%%%%%%%%%%%%%%%%%%%%%%%%%%%%%%%%%%%%%%%%%%%%%%%%%%%%%%%%%%%%%%%%%%%%%%%%%%

\section{Fundamental Types}
The fundamental respective primary types are exclusively modular types,
this means that an overflow or underrun of a variable does not lead to an
exception. If the result of an operation is not within the scope of a modular
type, $n:=2**Type'Size=Type'Last+1$ will be added to or subtracted for it
until the result is within the scope of the modular type. \\

\begin{lstlisting}{}
type Bit is mod 2 ** 1;
for Bit'Size use 1;

type Byte is mod 2 ** 8;
for Byte'Size use  8;

type Word is mod 2 ** 32;
for Word'Size use 32;

type DWord is mod 2 ** 64;
for DWord'Size use 64;
\end{lstlisting}

%%%%%%%%%%%%%%%%%%%%%%%%%%%%%%%%%%%%%%%%%%%%%%%%%%%%%%%%%%%%%%%%%%%%%%%%%%%
%%%%%%%%%%%%%%%%%%%%%%%%%%%%%%%%%%%%%%%%%%%%%%%%%%%%%%%%%%%%%%%%%%%%%%%%%%%


\subsubsection{Example}
\begin{lstlisting}[frame=brtl]{Modular Type}
with Ada.Text_IO;
with Crypto.Types;
procedure Example is
   use Crypto.Types;
   -- Byte has a scope of 0..255
   A, B : Byte;
begin
   A := 100;
   B := A + 250; -- Overflow at B
   A := A - 250; -- Underrun at A
   Ada.Text_IO.Put_Line("A: " &  A'IMG);
   Ada.Text_IO.Put_Line("B: " &  B'IMG);
end example;
\end{lstlisting}
\textit{Program output:\\}
\texttt{A:  106\\B:  94}

%%%%%%%%%%%%%%%%%%%%%%%%%%%%%%%%%%%%%%%%%%%%%%%%%%%%%%%%%%%%%%%%%%%%%%%%%%%
%%%%%%%%%%%%%%%%%%%%%%%%%%%%%%%%%%%%%%%%%%%%%%%%%%%%%%%%%%%%%%%%%%%%%%%%%%%
%%%%%%%%%%%%%%%%%%%%%%%%%%%%%%%%%%%%%%%%%%%%%%%%%%%%%%%%%%%%%%%%%%%%%%%%%%%


\subsection{Derived Types}
Derived types are types that are derived from elementary types. Regularly,
these are arrays, consisting of elementary types. \textbf{All non-private
arrays consisting of elementary types are interpreted as n-bit numbers within
the ACL}, where the first element (First) is considered to be the 
most-significant and the last element (Last) the least-significant element of 
the number. This is a fundamental characteristic of the ACL.

\begin{tabular}{p{\textwidth}}
\subsubsection{Bit}
\begin{lstlisting}{Bits}
  type Bits is array (Integer range <>) of Bit;
\end{lstlisting}\ \\ \ \\
\hline 
\end{tabular}


\begin{tabular}{p{\textwidth}}
\subsubsection{Bytes} 
\begin{lstlisting}{Bytes}
  type Bytes is array (Integer range <>) of Byte;
  
  subtype Byte_Word  is Bytes (0 .. 3);
  subtype Byte_DWord is Bytes (0 .. 7);

  subtype B_Block32  is Bytes (0 ..  3);
  subtype B_Block48  is Bytes (0 ..  5);
  subtype B_Block56  is Bytes (0 ..  6);
  subtype B_Block64  is Bytes (0 ..  7);
  subtype B_Block128 is Bytes (0 .. 15);
  subtype B_Block160 is Bytes (0 .. 19); 
  subtype B_Block192 is Bytes (0 .. 23);
  subtype B_Block256 is Bytes (0 .. 31);
\end{lstlisting}
A B\_BlockN always consists of a N bit array separated into N/8 Bytes.
The subtype B\_Block256 for example is a 256-bitstring, separated into
32 (258/8) byte blocks. In Ada, it can be represented by a byte array
consisting of 32 elements. \\ \ \\
\hline
\end{tabular}

%%%%%%%%%%%%%%%%%%%%%%%%%%%%%%%%%%%%%%%%%%%%%%%%%%%%%%%%%%%%%%%%%%%%%%%%%%%

\begin{tabular}{p{\textwidth}}
  \subsubsection{Words} 
  \begin{lstlisting}{Words}
  type Words   is array (Integer range <>) of Word;
  
  subtype W_Block128  is Words(0 .. 3);
  subtype W_Block160  is Words(0 .. 4);
  subtype W_Block192  is Words(0 .. 5);
  subtype W_Block256  is Words(0 .. 7);
  subtype W_Block512  is Words(0 .. 15);
  \end{lstlisting}
  A W\_BlockN always consists of a N bit array separated into N/32 Words.
  The subtype W\_Block256 for example is a 256-bitstring, separated into 8 (256/32) Word
  elements. In Ada, it can be represented by a Word array consisting of 8 elements.\\ \ \\
\hline
\end{tabular}

%%%%%%%%%%%%%%%%%%%%%%%%%%%%%%%%%%%%%%%%%%%%%%%%%%%%%%%%%%%%%%%%%%%%%%%%%%%
  
\begin{tabular}{p{\textwidth}}
  \subsubsection{DWords} 
  \begin{lstlisting}{DWords}
  type DWords   is array (Integer range <>) of DWord  
    
  subtype DW_Block256   is DWords(0 ..  3);
  subtype DW_Block384   is DWords(0 ..  5);
  subtype DW_Block512   is DWords(0 ..  7);
  subtype DW_Block1024  is DWords(0 .. 15);
  \end{lstlisting}
  A DW\_BlockN always consists of a N bit array separated into N/64 Words.
  The subtype DW\_Block256 for example is a 256-bitstring separated into 4 
  (256/64) DWord
  elements. In Ada, it can be represented by a DWord array with 4 elements.\\ 
\ \\
\hline
\end{tabular}

%%%%%%%%%%%%%%%%%%%%%%%%%%%%%%%%%%%%%%%%%%%%%%%%%%%%%%%%%%%%%%%%%%%%%%%%%%%
  
\begin{tabular}{p{\textwidth}}
  \subsubsection{Strings} 
  \begin{lstlisting}{Strings}
    subtype Hex_Byte  is String (1.. 2);
    subtype Hex_Word  is String (1.. 8);
    subtype Hex_DWord is String( 1..16);
  \end{lstlisting}\ \\
  \hline
\end{tabular}

%%%%%%%%%%%%%%%%%%%%%%%%%%%%%%%%%%%%%%%%%%%%%%%%%%%%%%%%%%%%%%%%%%%%%%%%%%%

\begin{tabular}{p{\textwidth}}
\subsubsection{Message blocks} 
\begin{lstlisting}{Message blocks}
subtype Message_Block_Length512  is Natural range 0 ..  63;
subtype Message_Block_Length1024 is Natural range 0 .. 127;

\end{lstlisting}
The Message\_Block\_Length types indicate the length of the actual message
stored within a 512 or 1024 bit message block M in characters (Bytes).
For example, splitting a 1152 bit message into 512 bit blocks results in
three 512 bit blocks. The actual message length of the last block is 16 
($1280 - 2 \cdot 512 = 128 / 8 = 16$).
The remaining 384 bits of the last message block are ''empty'', 
this means that they do not contain any part of the original message.
A variable of type M\_Length512 would be set to 32 in this case.
As seen above, these variable types are used for message block padding.
More information about padding can be found in the chapter about hash
functions (\ref{hash}).\\ \ \\
\hline 
\end{tabular}\ \\

%%%%%%%%%%%%%%%%%%%%%%%%%%%%%%%%%%%%%%%%%%%%%%%%%%%%%%%%%%%%%%%%%%%%%%%%%%%
%%%%%%%%%%%%%%%%%%%%%%%%%%%%%%%%%%%%%%%%%%%%%%%%%%%%%%%%%%%%%%%%%%%%%%%%%%%

\subsection{Bit-shifting Functions}
\begin{lstlisting}{Types-functions}
     function Shift_Left   (Value  : Byte;
                          Amount : Natural) 
			  return Byte;

   function Shift_Right  (Value  : Byte;
                          Amount : Natural) 
			  return Byte;

   function Rotate_Left  (Value  : Byte;
                          Amount : Natural)
			  return Byte;

   function Rotate_Right (Value  : Byte;
                          Amount : Natural)
			  return Byte;


   function Shift_Left   (Value : Word;
                          Amount : Natural) 
                          return Word;

   function Shift_Right  (Value  : Word; 
                          Amount : Natural) 
			  return Word;

   function Rotate_Left  (Value  : Word; 
                          Amount : Natural) 
			  return Word;

   function Rotate_Right (Value  : Word; 
                          Amount : Natural) 
			  return Word;


   function Shift_Left   (Value  : DWord; 
                          Amount : Natural) 
			  return DWord;

   function Shift_Right  (Value  : DWord;
                          Amount : Natural)
			  return DWord;

   function Rotate_Left  (Value  : DWord; 
                          Amount : Natural)
                           return DWord;

   function Rotate_Right (Value  : DWord;
                          Amount : Natural) 
			  return DWord;
\end{lstlisting}\ \\

%%%%%%%%%%%%%%%%%%%%%%%%%%%%%%%%%%%%%%%%%%%%%%%%%%%%%%%%%%%%%%%%%%%%%%%%%%%
%%%%%%%%%%%%%%%%%%%%%%%%%%%%%%%%%%%%%%%%%%%%%%%%%%%%%%%%%%%%%%%%%%%%%%%%%%%

\subsection{Basic Functions}\ 

\begin{tabular}{p{\textwidth}}
  \textbf{XOR}\ \\
  \begin{lstlisting}{}
    function "xor" (Left, Right : Bytes)   return Bytes;
    function "xor" (Left, Right : Words)   return Words;
    function "xor" (Left, Right : DWords) return DWords;
  \end{lstlisting}
  \underline{Prerequisite:}\\ Left'Length = Right'Length\\
  \underline{Exception:}\\
  Prerequisite violation :  Constraint\_Byte/Word/DWord\_Error\\ \ \\
  This functions perform a field-wise concatenation of the first two input 
  fields by
  using the XOR operation. Left(Left'First) is XOR concatenated with
  Right(Right'First) and Left(Left'Last) with Right(Right'Last).\\ \ \\
  \hline\\
\end{tabular}

%%%%%%%%%%%%%%%%%%%%%%%%%%%%%%%%%%%%%%%%%%%%%%%%%%%%%%%%%%%%%%%%%%%%%%%%%%%

\begin{tabular}{p{\textwidth}}
 \textbf{+}\ \\
\begin{lstlisting}{}
  function "+" (Left : Bytes; Right  : Byte)     return Bytes;
  function "+" (Left : Byte; Right   : Bytes)    return Bytes;
  function "+" (Left : Words; Right  : Word)     return Words;
  function "+" (Left : Word; Right   : Words)    return Words;
  function "+" (Left : Words; Right  : Byte)     return Words;
  function "+" (Left : DWords; Right : DWord)  return DWords;
  function "+" (Left : DWord;  Right : DWords) return DWords;
  function "+" (Left : DWords; Right : Byte)     return DWords;
\end{lstlisting}
\end{tabular}

\begin{tabular}{p{\textwidth}}
  This function interprets Left and Right as numbers where Left(Left'First)
  and Right(Right'First) denotes the least significant Byte/Word/DWord and
  Left(Left'Last) and Right(Right'Last) denotes the most significant
  Byte/Word/DWord of a number.
  The result of this function is the sum of the two numbers. \\ \ \\
  \textbf{Example:}
\begin{lstlisting}{}
procedure plus is 
  A : Byte := 200;
  B : Bytes(0..1) := (0 => 100, 1 => 16); 
  begin
  B := A+B; -- B := 2#11001000#+2#10000_01100100# 
  -- B(0) = 2#101100# = 44
  -- B(1) = 2#010001# = 17
end plus;
\end{lstlisting}
\end{tabular}

%%%%%%%%%%%%%%%%%%%%%%%%%%%%%%%%%%%%%%%%%%%%%%%%%%%%%%%%%%%%%%%%%%%%%%%%%%%
%%%%%%%%%%%%%%%%%%%%%%%%%%%%%%%%%%%%%%%%%%%%%%%%%%%%%%%%%%%%%%%%%%%%%%%%%%%

\subsection{Transformation Functions}\ \\

\subsubsection{ByteN}
\begin{lstlisting}{}
  function Byte0 (W : Word)  return Byte;
  function Byte1 (W : Word)  return Byte;
  function Byte2 (W : Word)  return Byte;
  function Byte3 (W : Word)  return Byte; 

  function Byte0 (D : DWord) return Byte;
  function Byte1 (D : DWord) return Byte;
  function Byte2 (D : DWord) return Byte;
  function Byte3 (D : DWord) return Byte;
  function Byte4 (D : DWord) return Byte;
  function Byte5 (D : DWord) return Byte;
  function Byte6 (D : DWord) return Byte;
  function Byte7 (D : DWord) return Byte;
\end{lstlisting}
\begin{tabular}{p{\textwidth}}
Let
$\mathtt{W : Word  \equiv B0||B1||B2||B3}$\\
and\\
$\mathtt{D : DWord \equiv B0||B1||B2||B3||B4||B5||B6||B7}$.\\ 
\ \\
Then the first function returns B0 of W,  the second B1 of W and so on.\\ \ \\
\hline\\
\end{tabular}

%%%%%%%%%%%%%%%%%%%%%%%%%%%%%%%%%%%%%%%%%%%%%%%%%%%%%%%%%%%%%%%%%%%%%%%%%%%

\begin{tabular}{p{\textwidth}}   
  \textbf{To\_Bytes}\ \\
  \begin{lstlisting}{}
    function To_Bytes(X : Word)  return Byte_Word;
    function To_Bytes(X : DWord) return Byte_DWord; 

    function To_Bytes(Word_Array  : Words)  return Bytes;
    function To_Bytes(DWord_Array : DWords) return Bytes;

    function To_Bytes(Message : String) return Bytes;
  \end{lstlisting}\\
  The first two functions convert a Word resp. DWord into a Byte\_Word resp.
  a Byte\_DWord.
  The most significant byte of X becomes the first element of the returned
  Byte array und the least significant byte of X becomes the last element of 
  the returned  Byte array.\\ \ \\
  \textbf{Example:}
  \begin{lstlisting}{}
    D : DWord := 16#AA_BB_CC_DD_EE_FF_11_22#;
    B : Byte_DWord := To_Bytes(D);
    -- B(0) := 16#AA#; B(1) := 16#BB#; B(2) := 16#CC#;
    -- B(3) := 16#DD#; B(4) := 16#EE#; B(5) := 16#FF#; 
    -- B(6) := 16#11#; B(7) := 16#22#;
  \end{lstlisting}\\
  The next two functions convert a Word or a DWord array into an Byte array.
  The first element of the input array becomes the first byte of the 
  returned byte array und the least significant byte of the last element 
  becomes the last element of the returned  Byte array.\\ \ \\
  The least function converts an ASCII string in a Byte array.
  For all $I \in$ Message'Range applies:
  B(I) =  Character'Pos(Message(I)).\\  \ \\
\hline\\
\end{tabular}


%%%%%%%%%%%%%%%%%%%%%%%%%%%%%%%%%%%%%%%%%%%%%%%%%%%%%%%%%%%%%%%%%%%%%%%%%%%

\begin{tabular}{p{\textwidth}}
  \textbf{R\_To\_Bytes}\ \\
  \begin{lstlisting}{}
    function R_To_Bytes (X :  Word) return Byte_Word; 
    function R_To_Bytes (X : DWord) return Byte_DWord;
  \end{lstlisting}
  This function transforms a Word resp. DWord into a Byte\_Word
  resp. Byte\_DWord.
  The the most significant Byte of X becomes the first element in the returned
  Byte array and the least  significant Byte of X becomes the last element in 
  the returned Byte array.\\ \ \\
  \hline\\
\end{tabular}

%%%%%%%%%%%%%%%%%%%%%%%%%%%%%%%%%%%%%%%%%%%%%%%%%%%%%%%%%%%%%%%%%%%%%%%%%%%

\begin{tabular}{p{\textwidth}}
  \textbf{To\_Word}\ \\
  \begin{lstlisting}{}
    function To_Word (X : Byte_Word)  return Word;

    function To_Word (A,B,C,D : Byte) return Word;

    function To_Word (A,B,C,D : Character) return Word;  
\end{lstlisting}\\ 
  The first function converts a  Byte\_Word into a Word.
  X(Byte\_Word'First) becomes the most significant Byte of the Word,
  X(Byte\_Word'Last) the last significant one.\\ \ \\
  The next function  transforms four Bytes (A, B, C, D) into a Word.
  A becomes the most significant Byte of the Word, D the last significant one.
  \\ \ \\
  The least function transforms four Characters (A, B, C, D) into a Word.
  The first Character (A'Pos) becomes the most significant byte of the
  resulting Word, the last character (D'Pos) the least significant one.\ \ \\
  \hline\\
\end{tabular}

%%%%%%%%%%%%%%%%%%%%%%%%%%%%%%%%%%%%%%%%%%%%%%%%%%%%%%%%%%%%%%%%%%%%%%%%%%%

\begin{tabular}{p{\textwidth}}   
  \textbf{R\_To\_Word}\ \\
  \begin{lstlisting}{}
    function R_To_Word (X : Byte_Word) return Word;   
  \end{lstlisting}
  This function transform a Byte\_Word into a Word.
  X(Byte\_Word'First) becomes the least significant Byte of the Word,
  X(Byte\_Word'Last) becomes the most significant Byte of the one.\\ \ \\
  \hline\\
\end{tabular}

%%%%%%%%%%%%%%%%%%%%%%%%%%%%%%%%%%%%%%%%%%%%%%%%%%%%%%%%%%%%%%%%%%%%%%%%%%%


\begin{tabular}{p{\textwidth}}
  \textbf{To\_Words}\ \\
  \begin{lstlisting}{}
    function To_Words(Byte_Array : Bytes)  return Words;
  \end{lstlisting}
  This function converts a Byte array into a Word array W
  Byte\_Array(Byte\_Array'First) becomes the most significant byte
  of W'First and the last element the Byte array becomes a byte of W'Last. 
  \textbf{Example:}\\
  \textbf{(i)}
\begin{lstlisting}{}
      B : Bytes(1..6) := (16#0A#, 16#14#, 16#1E#, 
                          16#28#, 16#32#, 16#3C#);
      W : Words := Byte_to_Dwords(B);
      -- W(D'First) = 16#0A_14_1E_28#
      -- W(D'Last)  = 16#32_3C_00_00#
    \end{lstlisting}\\ \ \\
  \textbf{(ii)}\\ \ \\
    \textit{Input:}$B=\underbrace{B(1)||B(2)||B(3)||B(4)}_{W(0)}
    \underbrace{B(5)||B(6)||B(7)||B(8)}_{W(1)}
    \underbrace{B(9)||B(10)}_{W(2)}$\\ \ \\
    \textit{Output:} $W=W(0)||W(1)||W(2)$\\ \ \\
    \hline\\
\end{tabular}

%%%%%%%%%%%%%%%%%%%%%%%%%%%%%%%%%%%%%%%%%%%%%%%%%%%%%%%%%%%%%%%%%%%%%%%%%%%


\begin{tabular}{p{\textwidth}}
  \textbf{To\_DWord}\ \\
  \begin{lstlisting}{} 
    function To_DWord (X : Byte_DWord) return DWord;
  \end{lstlisting}\\
  This function transforms a Byte\_DWord X into a DWord.
  X(Byte\_Word'First) becomes the most significant byte of the DWords,
  X(Byte\_Word'Last) becomes the least significant byte of the one.\\ \ \\
  \hline\\
\end{tabular}

%%%%%%%%%%%%%%%%%%%%%%%%%%%%%%%%%%%%%%%%%%%%%%%%%%%%%%%%%%%%%%%%%%%%%%%%%%%
  
\begin{tabular}{p{\textwidth}}   
\textbf{R\_To\_DWord}\ \\
  \begin{lstlisting}{}
    function R_To_DWord (X : Byte_DWord) return DWord;  
  \end{lstlisting}
  This function transforms a Byte\_DWord X into a DWord.
  X(Byte\_Word'First) becomes the least significant byte of the DWords,
  X(Byte\_Word'Last) becomes the most significant byte of the one.\\ \ \\
  \hline\\
\end{tabular}

%%%%%%%%%%%%%%%%%%%%%%%%%%%%%%%%%%%%%%%%%%%%%%%%%%%%%%%%%%%%%%%%%%%%%%%%%%%

\begin{tabular}{p{\textwidth}}
  \textbf{To\_DWords}\ \\
  \begin{lstlisting}{}
    function To_DWords  (Byte_Array : Bytes) return DWords;
  \end{lstlisting}
  This function transforms a byte array into a DWord array D.
  Byte\_Array(Byte\_Array'First) becomes the most significant byte
  of the D'First and the last element of the Byte array becomes a byte of
  D'Last. \\ \ \\
  \textbf{Example:}
  \begin{lstlisting}{}
    B : Bytes(1..10) : (10, 20, 30, 40, 50, 60, 70, 80, 90, 100);
    D : DWords := Byte_to_Dwords(B);
    -- D(D'First) =             16#0A_14_1E_28_32_3C_46_50#
    -- D(D'Last) = D(First+1) = 16#5A_64_00_00_00_00_00_00#
  \end{lstlisting}\\ \ \\
  \hline\\
\end{tabular}

%%%%%%%%%%%%%%%%%%%%%%%%%%%%%%%%%%%%%%%%%%%%%%%%%%%%%%%%%%%%%%%%%%%%%%%%%%%

\begin{tabular}{p{\textwidth}}
  \textbf{To\_String}\ \\
  \begin{lstlisting}{}
    function Bytes_To_String(ASCII : Bytes) return String;
  \end{lstlisting}
  This function transforms a Byte array into an String. Thereby
  every element of the Byte array will be interpreted as an ASCII-Code.\ \ \\
  \hline\\
\end{tabular}\ \\


%%%%%%%%%%%%%%%%%%%%%%%%%%%%%%%%%%%%%%%%%%%%%%%%%%%%%%%%%%%%%%%%%%%%%%%%%%%

\begin{tabular}{p{\textwidth}}
  \textbf{To\_Hex}\ \\
  \begin{lstlisting}{}
    function To_Hex(B : Byte)  return Hex_Byte;
    function To_Hex(W : Word)  return Hex_Word;
    function To_Hex(D : DWord) return Hex_DWord;
  \end{lstlisting}
  This functions  transforms a Byte, Word or DWord into an 2,8 or 16 byte
  hex string.
  For instance \texttt{Put(To\_Hex(W));} is equivalent to the C code
  \texttt{ printf(``\%08X'', i);}.\\ \ \\
  \textbf{Beispiel:}
\begin{lstlisting}{}
    B : Word := 0;
    W : Word := 16#AABBCCDDEEFF#;
    
    HB:Hex_Byte:=To_Hex(W); -- Es gilt: HB="00".
    HW:Hex_Word:=To_Hex(W); -- Es gilt: HW="0000AABBCCDDEEFF".
  \end{lstlisting}\ \\
\end{tabular}\ \\



%%%%%%%%%%%%%%%%%%%%%%%%%%%%%%%%%%%%%%%%%%%%%%%%%%%%%%%%%%%%%%%%%%%%%%%%%%%

\subsection{Functions}
\begin{lstlisting}{}
   function Is_Zero(Byte_Array  : Bytes)  return Boolean;
   function Is_Zero(Word_Array  : Words)  return Boolean;
   function Is_Zero(DWord_Array : DWords) return Boolean;
\end{lstlisting}
These functions return ``True'' if all fields of the array parameter
contain the value ``0''. Otherwise, ``False'' is returned.\\ \ \\

%%%%%%%%%%%%%%%%%%%%%%%%%%%%%%%%%%%%%%%%%%%%%%%%%%%%%%%%%%%%%%%%%%%%%%%%%%%
%%%%%%%%%%%%%%%%%%%%%%%%%%%%%%%%%%%%%%%%%%%%%%%%%%%%%%%%%%%%%%%%%%%%%%%%%%%

\subsection{Padding Procedures}
 \begin{lstlisting}{}
procedure Padding(Data            : in out Bytes;
                     Message_Length : in  Word;
                     Data2          : out Bytes);

procedure Padding(Data              : in  out Words;
                     Message_Length : in  Word;
                     Data2          : out Words);

 procedure Padding(Data              : in  out DWords;
                      Message_Length : in  Word;
                      Data2          : out DWords); 
 \end{lstlisting}
 \textbf{Parameters:}
 \textit{Data:}\\
 Data has to be an array containing a message.
 The message has to correspond to the following part of Data:\\
 Data(Data'First)..Data(Data'First+(Message\_Length-1))
\textit{Message\_Length:}\\
The length of the message stored in Data.
\textit{Data2:}\\
Is used for some special cases. Regularly
Is\_Zero(Data2) = true after calling this function.\\
\ \\
\textbf{Prerequisites:}
 \begin{enumerate}
 \item Data'Length = Data2'Length
 \item Message\_Length $\le$ Data'Length
 \item Word(Data'Length) - Message\_Length $>$ 2**16-1\footnotemark 
 \end{enumerate}
 \textbf{Postconditions:}
 \begin{enumerate}
 \item Message\_Length $<$ Data'Length-2 $\Rightarrow$ Is\_Zero(Data2) = True
 \item Is\_Zero(Data2) = False  $\Longleftrightarrow$ Message'Length $>$ 
   Data'Length-2
 \end{enumerate}
 \textbf{Exceptions:}\\
 \addtocounter{footnote}{-1} 
\begin{tabular}{l@{ : }l}
 Constraint\_Message\_Length\_Error &  Message\_Length $>$ Data'Length\\
 Constraint\_Length\_Error &  Data'Length $>$ (2 * Byte'Last)
 \footnote{Only if Data is an instance of type Byte}\\
 \end{tabular}\ \\ \ \\
 This functions add a number of null-Bytes/Words/Dwords to a message,
 followed by a Byte/Word/DWord representing the actual number of null-Bytes
 inserted. This might result in the following special cases:
 \begin{enumerate}
 \item Message\_Length+1=Data'Length\\
   After adding a null Byte to the message, no additional space is available
   for appending the actual number of Null-Bytes/Words/DWords inserted to the message
   as Data(Data'Last) has been filled up by the null-Byte/Word/DWord.
   In order to solve this problem, all fields of Data2 except for Data2'Length are set to 0.
   Data2'Last is set to Data2'Length.
 \item Message\_Length=Data'Length\\
   This case is treated analogous to the first special case with the following deviations:
   \begin{enumerate}
   \item No null-Byte/Word/DWord is appended to the message
   \item Data2'Last := Data2'Length-1
   \end{enumerate}
 \end{enumerate}

\subsubsection{Example}
\begin{lstlisting}{}
... 
  package BIO is new Ada.Text_Io.Modular_IO (Byte);

  X : Bytes(1..9) := (1 => 2, 2 => 4, 3 => 6, 
                      4 => 8, 5 => 9, others =>1);
  Y : Bytes(1..9);
begin
  Padding(X, N, Y);
  Put("X:");
  for I in X'Range loop
     Bio.Put(X(I));
  end loop;
  New_Line;
  if Is_Zero(Y) = False  then
      Put("Y:");
      for I in Y.all'Range loop
	 Bio.Put(Y.all(I));
      end loop;
   end if;
...
\end{lstlisting}\ \\
\underline{Output:}\\
\begin{itemize}
\item N = 5\\
  X : 2   4   6  8  9   0   0   0   3
\item N = 8\\
  X : 2   4   6   8   9   1   1   1   0\\
  Y : 0   0   0   0   0   0   0   0   9
\item N = 9\\
  X : 2   4   6   8   9   1   1   1   1\\
  Y : 0   0   0   0   0   0   0   0   8
\end{itemize}




