\part{Basics}
\chapter{Crypto.Types}
This package provides the fundamental and derived types with their
basic functionalities for the ACL.

\paragraph{IMPORTANT.} Applying in the ACL by importing the package:
\begin{lstlisting}
  with Crypto.Types;
\end{lstlisting}
%%%%%%%%%%%%%%%%%%%%%%%%%%%%%%%%%%%%%%%%%%%%%%%%%%%%%%%%%%
\section{Elemental Types}
The elemental or primary types here are only for modular types, i.e.,
overflow or underflow of a variable does not cause an exception. If
the result of an operation is not in the value range of the modular
type, then as long as $n:=2**$ Type'Size = Type'Last+1, the value will
be added or subtracted until the result is again in the range.
\begin{lstlisting}{}
  type Bit is mod 2;
  for  Bit'Size use 1;

  type Byte  is mod 2 ** 8;
  for  Byte'Size use 8;

  type DByte  is mod 2 ** 16;
  for  DByte'Size use 16;

  type Word is mod 2 ** 32;
  for  Word'Size use 32;
  type DWord is mod 2 ** 64;
  for  DWord'Size use 64;

  type Mod_Type is mod 2**32;
  for  Mod_Type'Size use 32;
\end{lstlisting}
The type \texttt{Mod\_Type} has the size of a CPU-word, in the ACL it
is 32 bits.

\subsubsection*{Example}
\begin{lstlisting}{Example}
  with Crypto.Types;
  with Ada.Text_IO;
  use Crypto.Types;
  procedure Example_Types is
    A, B : Byte;  -- Byte has a value range from 0 to 255;
  begin
    A := 100;
    B := A + 250; -- Overflow
    A := A - 250; -- Underflow
    Ada.Text_IO.Put_Line("A: " &  A'IMG);
    Ada.Text_IO.Put_Line("B: " &  B'IMG);
  end Example_Types;
\end{lstlisting}
Result of the program:
\begin{lstlisting}{Example}
A:  106
B:  94
\end{lstlisting}

%%%%%%%%%%%%%%%%%%%%%%%%%%%%%%%%%%%%%%%%%%%%%%%%%%%%%%%%
%%%%%%%%%%%%%%%%%%%%%%%%%%%%%%%%%%%%%%%%%%%%%%%%%%%%%%%%

\section{Derived Types}
Derived types are types, which are derived from the elementary
types. Regularly, these are arrays, consisting of elementary
types. \textbf{All non-private arrays consisting of elementary types
  are interpreted as n-bit numbers within the ACL}, where the first
element ('First) is considered to be the most significant and the last
element ('Last) the least significant element of the number. This is a
fundamental characteristic of the ACL.
\subsubsection*{Bit}
\begin{lstlisting}{Bits}
  type Bits is array (Integer range <>) of Bit;
\end{lstlisting}
\subsubsection*{Bytes}
\begin{lstlisting}{Bytes}
  type Bytes is array (Integer range <>) of Byte;
  subtype Byte_Word  is Bytes (0 .. 3);
  subtype Byte_DWord is Bytes (0 .. 7);

  type B_Block32  is Bytes (0 ..  3);
  type B_Block48  is Bytes (0 ..  5);
  type B_Block56  is Bytes (0 ..  6);
  type B_Block64  is Bytes (0 ..  7);
  type B_Block128 is Bytes (0 .. 15);
  type B_Block160 is Bytes (0 .. 19);
  type B_Block192 is Bytes (0 .. 23);
  type B_Block256 is Bytes (0 .. 31);
\end{lstlisting}
The type B\_BlockN consists of a n-bit array, separated into n/8
bytes, e.g., the type \texttt{B\_Block256} is a 256-bit string,
separated into an array of 32 bytes. In Ada, it can be also
represented as a byte array consisting of 32 elements.

\hhline
%%%%%%%%%%%%%%%%%%%%%%%%%%%%%%%%%%%%%%%%%%%%%%%%%%%%%%%%%%%%%%%%%%%%%%%%%%%
\subsubsection*{Words}
\begin{lstlisting}{Words}
  type Words is array (Integer range <>) of Word;
  
  type W_Block128  is Words(0 ..  3);
  type W_Block160  is Words(0 ..  4);
  type W_Block192  is Words(0 ..  5);
  type W_Block256  is Words(0 ..  7);
  type W_Block512  is Words(0 .. 15);
\end{lstlisting}
The type W\_BlockN consists of a n-bit array, separated into n/32
words, e.g., the type \texttt{W\_Block256} is a 256-bit string,
separated into an array of 8 words. In Ada, it can be also represented
as a word array consisting of 8 elements.

\hhline
%%%%%%%%%%%%%%%%%%%%%%%%%%%%%%%%%%%%%%%%%%%%%%%%%%%%%%%%%%%%%%%%%%%%%%%%%%%
\subsubsection*{DWords}
\begin{lstlisting}{DWords}
  type DWords is array (Integer range <>) of DWord;
  
  type DW_Block128   is DWords(0 ..  1);
  type DW_Block256   is DWords(0 ..  3);
  type DW_Block384   is DWords(0 ..  5);
  type DW_Block512   is DWords(0 ..  7);
  type DW_Block1024  is DWords(0 .. 15);
\end{lstlisting}
The type DW\_BlockN consists of a n-bit array, separated into n/64
DWords, e.g., the type \texttt{DW\_Block256} is a 256-bit string,
separated into 4 DWords. In Ada, it can be represented as a DWord
array of 4 elements.

\hhline
%%%%%%%%%%%%%%%%%%%%%%%%%%%%%%%%%%%%%%%%%%%%%%%%%%%%%%%%%%%%%%%%%%%%%%%%%%%

\subsubsection*{Strings}
\begin{lstlisting}{Strings}
  subtype Hex_Byte  is String (1..  2);
  subtype Hex_Word  is String (1..  8);
  subtype Hex_DWord is String (1.. 16);
\end{lstlisting}
The subtype \texttt{Hex\_Byte} is a string, which consists of 2
characters. It can be used to convert a byte to a string in
hexadecimal. The subtype \texttt{Hex\_Word} is used to convert a word
to a string in hexadecimal, and \texttt{Hex\_DWord} is used for a
dword.

\hhline
%%%%%%%%%%%%%%%%%%%%%%%%%%%%%%%%%%%%%%%%%%%%%%%%%%%%%%%%%%%%%%%%%%%%%%%%%%%
\subsubsection*{Message Block Length}
\begin{lstlisting}{Message blocks}
  subtype Message_Block_Length512  is Natural range 0 ..  64;
  subtype Message_Block_Length1024 is Natural range 0 .. 128;
\end{lstlisting}
The two Message\_Block\_Length types indicate the length of the actual
message stored within a message block in bytes. For example, splitting
a 1152-bit message by 512-bit blocks results in three 512-bit
blocks. The actual message length of the last block is 16 ($1152 - 2
\cdot 512 = 128$ bits $= 16$ bytes).  The remaining 384 bits of the
last message block are ''empty'', which means that they do not contain
any part of the original message. These two types are used for message
block padding.  More information about padding can be found in Chapter
\ref{Hash}.

%%%%%%%%%%%%%%%%%%%%%%%%%%%%%%%%%%%%%%%%%%%%%%%%%%%%%%%%%%%%%%%%%%%%%%%%%%%
%%%%%%%%%%%%%%%%%%%%%%%%%%%%%%%%%%%%%%%%%%%%%%%%%%%%%%%%%%%%%%%%%%%%%%%%%%%

\section{Functions and Procedures}
\subsubsection{Bit-Shifting Functions}
\begin{lstlisting}{}
   function Shift_Left  (Value : Natural;
                		    Amount: Natural) return Natural;
   function Shift_Right (Value : Natural;
   						    Amount: Natural) return Natural;
   function Shift_Left  (Value : Byte; Amount : Natural) return Byte;
   function Shift_Right (Value : Byte; Amount : Natural) return Byte;
   function Rotate_Left (Value : Byte; Amount : Natural) return Byte;
   function Rotate_Right(Value : Byte; Amount : Natural) return Byte;
\end{lstlisting}
These functions are intrinsic. They are used to make a transformation
of the value. They can be also applied on values of type
\texttt{DByte, Word, DWord} and \texttt{Mod\_Type}.

The transformations are made on bit positions. For example, calling
the function \texttt{Shift\_Left()} on a value of type \texttt{Word}
with amount 2 causes the value being shifted two bit positions to left
and padded with two zeros at the end.\\
\begin{lstlisting}{}
   function Shift_Block_Left (B_Block: B_Block128; Amount : Natural) return B_Block128;
   function Shift_Block_Right(B_Block: B_Block128; Amount : Natural) return B_Block128;
\end{lstlisting}
The two functions are used to shift a 128-bit block in a bit position
specified by the \texttt{Amount}.

\hhline
%%%%%%%%%%%%%%%%%%%%%%%%%%%%%%%%%%%%%%%%%%%%%%%%%%%%%%%%%%%%%%%%%%%%%%%%%%%
%%%%%%%%%%%%%%%%%%%%%%%%%%%%%%%%%%%%%%%%%%%%%%%%%%%%%%%%%%%%%%%%%%%%%%%%%%%
\subsubsection*{XOR}
\begin{lstlisting}{}
  function "xor" (Left, Right : Bytes)        return Bytes;
  function "xor" (Left: Bytes ; Right : Byte) return Bytes;
  function "xor" (Left, Right : Words)        return Words;
  function "xor" (Left, Right : DWords)       return DWords;
\end{lstlisting}
These functions perform a field-wise concatenation of the two input
fields by using the XOR operation. Left(Left'First) is XORed with
Right(Right'First) till Left(Left'Last) with Right(Right'Last). Note,
in the second function the Right is XORed with Left(Left'Last).\\

\paragraph{Exceptions.} If the lengths of the two values are not
equal, except the second function:\quad\texttt{Constraint\_Bytes\_Error}\,,
\texttt{Constraint\_Words\_Error} or \texttt{Constraint\_DWords\_Error}.

\begin{lstlisting}{}
  function "xor"(Left, Right : B_Block64)    return   B_Block64;
  function "xor"(Left, Right : B_Block128)   return   B_Block128;
  function "xor"(Left, Right : W_Block512)   return   W_Block512;
  function "xor"(Left, Right : DW_Block512)  return   DW_Block512;
  function "xor"(Left, Right : DW_Block1024) return   DW_Block1024;
\end{lstlisting}
The XOR operations on blocks are needed for generic packages to
convert a specific block.

\hhline
%%%%%%%%%%%%%%%%%%%%%%%%%%%%%%%%%%%%%%%%%%%%%%%%%%%%%%%%%%%%%%%%%%%%%%%%%%%
\subsubsection*{AND}
\begin{lstlisting}{}
 function "and"(Left, Right : Bytes) return Bytes;
\end{lstlisting}
The function is applied to make bit-wise AND of two values of equal
length.\\

\paragraph{Exception.} If the lengths of the two values are not
equal:\quad\texttt{Constraint\_Bytes\_Error}.

\hhline
%%%%%%%%%%%%%%%%%%%%%%%%%%%%%%%%%%%%%%%%%%%%%%%%%%%%%%%%%%%%%%%%%%%%%%%%%%%
\subsubsection*{+}
\begin{lstlisting}{}
  function "+" (Left : Bytes;      Right : Byte)   return Bytes;
  function "+" (Left : Byte;       Right : Bytes)  return Bytes;
  function "+" (Left : Words;      Right : Word)   return Words;
  function "+" (Left : Word;       Right : Words)  return Words;
  function "+" (Left : Words;      Right : Byte)   return Words;
  function "+" (Left : DWords;     Right : DWord)  return DWords;
  function "+" (Left : DWord;      Right : DWords) return DWords;
  function "+" (Left : DWords;     Right : Byte)   return DWords;
  function "+" (Left : B_Block128; Right : Byte)   return B_Block128;
\end{lstlisting}
The "+" function adds values in byte positions.  It calculates from
the least significant ('Last) value to the most significant ('First)
value, i.e., from right to left.\\ \textbf{Example:}
\begin{lstlisting}{}
  procedure Example_Add is
 	  A : Byte := 200;
 	  B : Bytes(0..1) := (0 => 100, 1 => 116);
  begin
  	  B := A + B;  -- B   := 2#11001000# + 2#01100100_01110100#
                  -- B(0) = 2#01100100# = 100
                  -- B(1) = 2#00111100# = 60 -- Overflow
  end Example_Add;
\end{lstlisting}

\hhline
%%%%%%%%%%%%%%%%%%%%%%%%%%%%%%%%%%%%%%%%%%%%%%%%%%%%%%%%%%%%%%%%%%%%%%%%%%%
%%%%%%%%%%%%%%%%%%%%%%%%%%%%%%%%%%%%%%%%%%%%%%%%%%%%%%%%%%%%%%%%%%%%%%%%%%
\subsubsection*{ByteN}
\begin{lstlisting}{}
  function Byte0 (W : Word)  return Byte;
  function Byte1 (W : Word)  return Byte;
  function Byte2 (W : Word)  return Byte;
  function Byte3 (W : Word)  return Byte;

  function Byte0 (D : DWord) return Byte;
  function Byte1 (D : DWord) return Byte;
  function Byte2 (D : DWord) return Byte;
  function Byte3 (D : DWord) return Byte;
  function Byte4 (D : DWord) return Byte;
  function Byte5 (D : DWord) return Byte;
  function Byte6 (D : DWord) return Byte;
  function Byte7 (D : DWord) return Byte;
\end{lstlisting}
Let W $\mathtt{: Word  := B0||B1||B2||B3}$,\\
and D $\mathtt{: DWord := B0||B1||B2||B3||B4||B5||B6||B7}$.\\
Then, \texttt{B0} is the most significant byte, and \texttt{B3} of W
and \texttt{B7} of D are the least significant bytes, respectively.
The first function returns B0 of W, the second B1 of W and so on.

\hhline
%%%%%%%%%%%%%%%%%%%%%%%%%%%%%%%%%%%%%%%%%%%%%%%%%%%%%%%%%%%%%%%%%%%%%%%%%%%
\subsubsection*{To\_Bytes}
\begin{lstlisting}{}
  function To_Bytes(X           : Word)   return Byte_Word;
  function To_Bytes(X           : DWord)  return Byte_DWord;
  function To_Bytes(Word_Array  : Words)  return Bytes;
  function To_Bytes(DWord_Array : DWords) return Bytes;
  function To_Bytes(Message     : String) return Bytes;
\end{lstlisting}
These functions convert the input value of different types into a byte
array. The most significant byte of the first element of the input
array becomes the first byte of the returned byte array und the least
significant byte of the last element becomes the last byte of the
returned byte array. For a message in string, it is converted as ASCII
codes stored in a byte array.

\paragraph{Example:}
\begin{lstlisting}{}
  D : DWord      := 16#AA_BB_CC_DD_EE_FF_11_22#;
  B : Byte_DWord := To_Bytes(D);
      -- B(0) := 16#AA#; B(1) := 16#BB#; B(2) := 16#CC#;
      -- B(3) := 16#DD#; B(4) := 16#EE#; B(5) := 16#FF#;
      -- B(6) := 16#11#; B(7) := 16#22#;
\end{lstlisting}
\begin{lstlisting}
  function To_Bytes(B : B_Block64)   return Bytes;
  function To_Bytes(B : B_Block128)  return Bytes;
  function To_Bytes(B : B_Block192)  return Bytes;
  function To_Bytes(B : B_Block256)  return Bytes;
  function To_Bytes(W : W_Block160)  return Bytes;
  function To_Bytes(W : W_Block256)  return Bytes;
  function To_Bytes(W : W_Block512)  return Bytes;
  function To_Bytes(D : DW_Block512) return Bytes;
\end{lstlisting}
These functions convert blocks to bytes in the same principle, they
are needed for generic packages to convert a specific block.

\hhline

%%%%%%%%%%%%%%%%%%%%%%%%%%%%%%%%%%%%%%%%%%%%%%%%%%%%%%%%%%%%%%%%%%%%%%%%%%%
\subsubsection*{R\_To\_Bytes}
\begin{lstlisting}{}
  function R_To_Bytes (X : Word)  return Byte_Word;
  function R_To_Bytes (X : DWord) return Byte_DWord;
\end{lstlisting}
It transforms a Word resp. DWord into a Byte\_Word resp. Byte\_DWord
in reverse order. The most significant byte of X becomes the last
element in the returned byte array and the least significant byte of X
becomes the first element in the returned byte array.

\hhline
%%%%%%%%%%%%%%%%%%%%%%%%%%%%%%%%%%%%%%%%%%%%%%%%%%%%%%%%%%%%%%%%%%%%%%%%%%
\subsubsection*{To\_Word vs. R\_To\_Word}
\begin{lstlisting}{}
  function To_Word (X       : Byte_Word) return Word;
  function To_Word (A,B,C,D : Byte)      return Word;
  function To_Word (A,B,C,D : Character) return Word;

  function R_To_Word (X : Byte_Word)     return Word;
\end{lstlisting}
The first function converts a Byte\_Word into a Word. X'First becomes
the most significant byte of the Word, and X'Last is the least
significant one.  It can transform four bytes (A, B, C, D) to make a
word. Then, A becomes the most significant byte of the word, and D the
least significant byte.  It can also transform four characters (A, B,
C, D) to make a word. The first character (A'Pos) becomes the most
significant byte of the resulted word, and the last character (D'Pos)
the least significant one.

The function \texttt{R\_To\_Word()} makes a word from the Byte\_Word
in reverse order. X'First becomes the least significant byte of the
word, and X'Last becomes the most significant byte of the
word.

\hhline
%%%%%%%%%%%%%%%%%%%%%%%%%%%%%%%%%%%%%%%%%%%%%%%%%%%%%%%%%%%%%%%%%%%%%%%%%%%
\subsubsection*{To\_Words}
\begin{lstlisting}{}
  function To_Words(Byte_Array : Bytes) return Words;
\end{lstlisting}
This function converts a byte array into a word array
(Word\_Array). Byte\_Array'First becomes the most significant byte of
Word\_Array'First and Byte\_Array'Last becomes the least significant
byte of Word\_Array'Last.

\paragraph{Example:}
\begin{lstlisting}{}
  B : Bytes(1..6):= (16#0A#, 16#14#, 16#1E#, 16#28#, 16#32#, 16#3C#);
  W : Words := To_Words(B);
      -- W(D'First) = 16#0A_14_1E_28#
      -- W(D'Last)  = 16#32_3C_00_00# Zero padding
\end{lstlisting}
Input: B $:=\underbrace{B(1)||B(2)||B(3)||B(4)}_{W(0)}
                    \underbrace{B(5)||B(6)}_{W(1)}$\,,\\ \ \\
Output: W $:=W(0)||W(1)$.

\hhline
%%%%%%%%%%%%%%%%%%%%%%%%%%%%%%%%%%%%%%%%%%%%%%%%%%%%%%%%%%%%%%%%%%%%%%%%%%%
\subsubsection*{To\_DWord vs. R\_To\_DWord}
\begin{lstlisting}{}
  function To_DWord   (X : Byte_DWord) return DWord;
  function R_To_DWord (X : Byte_DWord) return DWord;
\end{lstlisting}
The function \texttt{To\_DWord()} transforms a value of Byte\_DWord
into a DWord.  X'First becomes the most significant byte of the
resulted DWord, X'Last becomes the least significant byte.  While the
function \texttt{R\_To\_DWord()} transforms a Byte\_DWord into a DWord
in reverse order. X'First becomes the least significant byte of the
resulted DWord, and X'Last becomes the most significant byte.

\hhline
%%%%%%%%%%%%%%%%%%%%%%%%%%%%%%%%%%%%%%%%%%%%%%%%%%%%%%%%%%%%%%%%%%%%%%%%%%%
\subsubsection*{To\_DWords}
\begin{lstlisting}{}
  function To_DWords (Byte_Array : Bytes) return DWords;
\end{lstlisting}
It transforms a byte array into a DWord array
(DWord\_Array). Byte\_Array'First becomes the most significant byte of
the DWord\_Array'First, and Byte\_Array'Last becomes the least
significant byte of DWord\_Array'Last.

\paragraph{Example:}
\begin{lstlisting}{}
  B : Bytes(1..10) := (16#0A#, 16#14#, 16#1E#, 16#28#, 16#32#,
                       16#3C#, 16#46#, 16#50#, 16#5A#, 16#64#);
  D : DWords := To_DWords(B);
      -- D(D'First) = 16#0A_14_1E_28_32_3C_46_50#
      -- D(D'Last)  = 16#5A_64_00_00_00_00_00_00#
\end{lstlisting}
\hhline
%%%%%%%%%%%%%%%%%%%%%%%%%%%%%%%%%%%%%%%%%%%%%%%%%%%%%%%%%%%%%%%%%%%%%%%%%%%
\subsubsection*{To\_String}
\begin{lstlisting}{}
  function To_String(ASCII : Bytes) return String;
\end{lstlisting}
It transforms a byte array to a string. Thereby every element of the
byte array will be interpreted as an ASCII-Code.

\hhline
%%%%%%%%%%%%%%%%%%%%%%%%%%%%%%%%%%%%%%%%%%%%%%%%%%%%%%%%%%%%%%%%%%%%%%%%%%%
\subsubsection*{To\_Hex}
\begin{lstlisting}{}
  function To_Hex(B : Byte)  return Hex_Byte;
  function To_Hex(W : Word)  return Hex_Word;
  function To_Hex(D : DWord) return Hex_DWord;
\end{lstlisting}
These functions transform values of Byte, Word or DWord to strings,
which display in hex form.

\paragraph{Example:}
\begin{lstlisting}{}
  B : Word  := 0;
  W : DWord := 16#AA_BB_CC_DD_EE_FF#;
  HB:Hex_Word :=To_Hex(B);  -- HB="00_00_00_00"
  HW:Hex_DWord:=To_Hex(W);  -- HW="00_00_AA_BB_CC_DD_EE_FF"
\end{lstlisting}

\hhline
%%%%%%%%%%%%%%%%%%%%%%%%%%%%%%%%%%%%%%%%%%%%%%%%%%%%%%%%%%%%%%%%%%%%%%%%%%%
\subsubsection*{To\_Block}
\begin{lstlisting}{}
  function To_B_Block64 (B : Bytes) return B_Block64;
  function To_B_Block128(B : Bytes) return B_Block128;
  function To_B_Block192(B : Bytes) return B_Block192;
  function To_B_Block256(B : Bytes) return B_Block256;
\end{lstlisting}
These functions transform bytes to certain block types.

\hhline
%%%%%%%%%%%%%%%%%%%%%%%%%%%%%%%%%%%%%%%%%%%%%%%%%%%%%%%%%%%%%%%%%%%%%%%%%%%
\subsubsection{Is\_Zero}
\begin{lstlisting}{}
  function Is_Zero(Byte_Array  : Bytes)  return Boolean;
  function Is_Zero(Word_Array  : Words)  return Boolean;
  function Is_Zero(DWord_Array : DWords) return Boolean;
\end{lstlisting}
These functions return ''True'' if all fields of the array parameter
contain value ''0'', otherwise, ''False'' is returned.

\hhline
%%%%%%%%%%%%%%%%%%%%%%%%%%%%%%%%%%%%%%%%%%%%%%%%%%%%%%%%%%%%%%%%%%%%%%%%%%%
%%%%%%%%%%%%%%%%%%%%%%%%%%%%%%%%%%%%%%%%%%%%%%%%%%%%%%%%%%%%%%%%%%%%%%%%%%%
\subsubsection{Padding}
\begin{lstlisting}{}
   procedure Padding(Data           : in  out Bytes;
                     Message_Length : in      Word;
                     Data2          : out     Bytes);
   procedure Padding(Data           : in  out Words;
                     Message_Length : in      Word;
                     Data2          : out     Words);
   procedure Padding(Data           : in  out DWords;
                     Message_Length : in      Word;
                     Data2          : out     DWords);
\end{lstlisting}
The padding procedures fill the data with zero bytes/words/dwords,
together with a counter of the padded zero bytes/words/dwor
ds. \texttt{Data} is the message to be padded, and the term
\texttt{Message\_Length} specifies the actual length of the
message. The message corresponds to the following range of
\texttt{Data}: \texttt{Data(Data'First)..Data(Data'First+Message\_Length-1)}. \\
\texttt{Data2} is used in special situations, normally \texttt{Is\_Zero(Data2)} = True.

\paragraph{Exceptions:}
\texttt{Message\_Length} is greater than the length of
\texttt{Data}:\quad\texttt{Constraint\_Message\_Length\-\_Error}.\\ If
the lengths of \texttt{Data} and \texttt{Data2} are not
equal:\quad\texttt{Constraint\_Length\_Error}.\\ \ \\ 
\texttt{Data2} is used in the following two situations:
\begin{itemize}
\item Message\_Length = Data'Length:\quad no zero bytes/words/dwords
  will be padded to \texttt{Data}, and Data2'Last := Data'Length-1\,;
\item Message\_Length+1 = Data'Length:\quad only one zero
  byte/word/dword is padded, and there is no more space for the
  counter. To solve the problem, Data2 is initialized with zero
  bytes/words/dwords, and Data2'Last := Data'Length.
\end{itemize}
\textbf{Example:}
\begin{lstlisting}{}
  procedure Example_Padding is
    X: Bytes(1..9) := (1=>2, 2=>4, 3=>6, 4=>8, 5=>9, others=>1);
    Y: Bytes(1..9);
  begin
    Padding(X,5,Y);
    for I in X'Range loop
        Put(X(I)'Img);
    end loop;
    New_Line;
    for I in Y'Range loop
        Put(Y(I)'Img);
    end loop;
  end Example_Padding;
\end{lstlisting}
\begin{itemize}
\item N = 5\\
\qquad X: 2 4 6 8 9 0 0 0 3
\item N = 8\\
\qquad X: 2 4 6 8 9 1 1 1 0\\
\qquad Y: 0 0 0 0 0 0 0 0 9
\item N = 9\\
\qquad X: 2 4 6 8 9 1 1 1 1\\
\qquad Y: 0 0 0 0 0 0 0 0 8
\end{itemize}
