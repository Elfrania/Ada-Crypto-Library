\chapter{Crypto.Types.Big\_Numbers}
The package provides a generic type \texttt{Big\_Unsigned} with a set
of procedures and functions. This type uses internally an array, which
consists of 32-bit words. And this array is as a modular type
interpreted. For this reason, it is only possible to generate
variables of type \texttt{Big\_Unsigned} with that, their bit length
is a multiple of 32 bits. Since the package works without pointer and
inline assembler, it is much slower than e.g. for efficiency optimized
MPI (multi-precision-integers) of
GnuPG\footnote{http://www.gnupg.org/documentation/manuals/gcrypt/MPI-library.html,
  November 2012}. This big number package is originally based on the
Big Number library\footnote{http://bignumber.chez.com/, November 2012}
by Jerome Delcourt, which should be used here. But it turned out that
the library didn't meet some desired requirements. So this package
would be completely rewritten. A significant contribution to the
current code is the analysis of the source code from
\texttt{java.math.BigInteger}
\footnote{http://docs.oracle.com/javase/1.4.2/docs/api/java/math/BigInteger.html,
  November 2012} and
BeeCrypt\footnote{http://sourceforge.net/projects/beecrypt/, November
  2012} library by Bob Deblier.% and also the following codes.  The
package is specified in Crypto.Types.Big\_Numbers.

\section{API}
\subsubsection*{Generic Part}
\begin{lstlisting}{}
  generic
    Size : Positive;
\end{lstlisting}
\paragraph{Exception:} If Size $\not= k \cdot 32,\; k \in N$:\quad
\texttt{Constraint\_Size\_Error}.
\newpage

\subsubsection*{Types}
\begin{lstlisting}{}
  type Big_Unsigned is private;
\end{lstlisting}
It is the basic type in the package. \texttt{Big\_Unsigned} represents
a modular number of n ($n = k \cdot 32$) bits. A variable of this type
will be always initialized with a constant named
\texttt{Big\_Unsigned\_Zero}.

\hhline
\begin{lstlisting}{}
  subtype Number_Base is Integer range 2 .. 16;
\end{lstlisting}
The type will be needed later for the conversion from a
\texttt{Big\_Unsigned} to a string. A \texttt{Big\_Unsigned} variable
can be represented as a number based on a base from this
subtype.

\hhline
\subsubsection*{Constants}
\begin{lstlisting}{}
  Big_Unsigned_Zero   : constant Big_Unsigned; -- 0
  Big_Unsigned_One    : constant Big_Unsigned; -- 1
  Big_Unsigned_Two    : constant Big_Unsigned; -- 2
  Big_Unsigned_Three  : constant Big_Unsigned; -- 3
  Big_Unsigned_Four   : constant Big_Unsigned; -- 4
  Big_Unsigned_Ten    : constant Big_Unsigned; -- 10
  Big_Unsigned_Sixteen: constant Big_Unsigned; -- 16
  Big_Unsigned_First  : constant Big_Unsigned; -- 0
  Big_Unsigned_Last   : constant Big_Unsigned; -- "Big_Unsigned'Last"
\end{lstlisting}

%%%%%%%%%%%%%%%%%%%%%%%%%%%%%%%%%%%%%%%%%%%%%%%%%%%%%%%%%%%%%%%%%
%%%%%%%%%%%%%%%%%%%%%%%%%%%%%%%%%%%%%%%%%%%%%%%%%%%%%%%%%%%%%%%%%
\subsubsection*{Comparison Operations}
\begin{itemize}
\item Between two values of the same type \texttt{Big\_Unsigned}:
\end{itemize}
\begin{lstlisting}{}
  function "=" (Left, Right : Big_Unsigned) return Boolean;
  function "<" (Left, Right : Big_Unsigned) return Boolean;
  function ">" (Left, Right : Big_Unsigned) return Boolean;
  function "<="(Left, Right : Big_Unsigned) return Boolean;
  function ">="(Left, Right : Big_Unsigned) return Boolean;

  function Min(X, Y : in Big_Unsigned) return Big_Unsigned;
  function Max(X, Y : in Big_Unsigned) return Big_Unsigned;
\end{lstlisting}
\begin{itemize}
\item Between one value of type \texttt{Big\_Unsigned} and one value
  of type \texttt{Mod\_Type}, also vice versa.
\end{itemize}
\begin{lstlisting}{}
  function "="(Left : Big_Unsigned; Right : Mod_Type) return Boolean;
  function "<"(Left : Big_Unsigned; Right : Mod_Type) return Boolean;
  function ">"(Left : Big_Unsigned; Right : Mod_Type) return Boolean;
  function "<="(Left: Big_Unsigned; Right : Mod_Type) return Boolean;
  function ">="(Left: Big_Unsigned; Right : Mod_Type) return Boolean;
\end{lstlisting}
\subsubsection*{Basic Operations}
The basic functions can be used to operate values of
\texttt{Big\_Unsigned} and \texttt{Mod\_Type}. Additionally, the
functions $"+","-","*","/"$ can be applied on two values, which are
both of type \texttt{Big\_Unsigned}.

\begin{lstlisting}{}
  function "+"(Left:Big_Unsigned;Right:Mod_Type) return Big_Unsigned;
  function "-"(Left:Mod_Type;Right:Big_Unsigned) return Big_Unsigned;
  function "*"(Left:Big_Unsigned;Right:Mod_Type) return Big_Unsigned;
  function "/"(Left:Mod_Type;Right:Big_Unsigned) return Big_Unsigned;

  function "xor"(Left, Right : Big_Unsigned) return Big_Unsigned;
  function "or" (Left, Right : Big_Unsigned) return Big_Unsigned;
  function "and"(Left, Right : Big_Unsigned) return Big_Unsigned;
  function "and"(Left  : Big_Unsigned;
  					  Right : Mod_Type) return Big_Unsigned;
  function "and"(Left  : Mod_Type;
  					  Right : Big_Unsigned) return Big_Unsigned;
  function "**" (Left, Right : Big_Unsigned) return Big_Unsigned;
  function "mod"(Left, Right : Big_Unsigned) return Big_Unsigned;
  function "mod"(Left  : Big_Unsigned;
  					  Right : Mod_Type) return Big_Unsigned;
\end{lstlisting}

%%%%%%%%%%%%%%%%%%%%%%%%%%%%%%%%%%%%%%%%%%%%%%%%%%%%%%%%%%%%%%%%%%

\subsubsection*{Multiplication}
In addition to the elementary multiplication functions, further
multiplication algorithms are implemented to develop their potentials
in \texttt{Big\_Unsigned}.\\

\begin{lstlisting}{}
  function Russ(Left, Right: Big_Unsigned) return Big_Unsigned;
\end{lstlisting}
The Russian multiplication is a systematic method, which does not
require the multiplication table, only the ability to multiply and
divide by 2, and to add, details can be found in \cite{Russ}.

\hhline

\begin{lstlisting}{}
  function Karatsuba(Left, Right : Big_Unsigned) return Big_Unsigned;
  function Karatsuba_P(Left,Right: Big_Unsigned) return Big_Unsigned;
\end{lstlisting}
The Karatsuba algorithm can reduce the multiplication of two n-digit
numbers to at most $3 n^{\log_2 3}\approx 3 n^{1.585}$ single-digit
multiplications usually (and exactly $n^{\log_2 3}$ when n is a power
of 2) \cite{DBLP:reference/crypt/2011}. In parallel Karatsuba
function, the product is calculated in tasks.

\hhline
\begin{lstlisting}{}
  function Toom_Cook(Left, Right : Big_Unsigned) return Big_Unsigned;
  function Toom_Cook_P(Left,Right: Big_Unsigned) return Big_Unsigned;
\end{lstlisting}
The Toom-Cook multiplication splits up two large integers $a$ and $b$
into $k$ smaller parts (each of length $l$ ), and performs operations
on the parts. As $k$ grows, the multiplication sub-operations may be
combined, thus the overall complexity can be reduced, and the
multiplication sub-operations can be computed recursively using
Toom-Cook multiplication again, and so on
\cite{DBLP:reference/crypt/2011}.

\paragraph{Remarks.} All algorithms can be called explicitly and used
for multiplication of two \texttt{Big\_Unsigned} values.  A
performance improvement can then be achieved by combining different
algorithms, which contain different strengths. In the ACL, by calling
the elementary multiplication (function $'*'$) the sum of two
parameters' bit length is compared to the length of
\texttt{Mod\_Type}. The elementary multiplication is used for small
factors, for large bit length (between 2800 bits and 3600 bits) the
parallel Karatsuba (function \texttt{Karatsuba\_P()}) will be called,
and for the length, which is longer than 3600 bits, the parallel
Toom\_Cook algorithm (function \texttt{Toom\_Cook\_P()}) will come in
use.
%%%%%%%%%%%%%%%%%%%%%%%%%%%%%%%%%%%%%%%%%%%%%%%%%%%%%%%%%%%%%%%%%
%%%%%%%%%%%%%%%%%%%%%%%%%%%%%%%%%%%%%%%%%%%%%%%%%%%%%%%%%%%%%%%%%
\section{Utils}
\texttt{Crypto.Types.Big\_Numbers.Utils} is a separate body of
\texttt{Crypto.Types.Big\_Numbers}. Many useful functions and
procedures are provided.
\begin{lstlisting}{}
  procedure Swap(X, Y : in out Big_Unsigned);
\end{lstlisting}
The two parameters are exchanged.

\hhline
\begin{lstlisting}{}
  procedure Set_Least_Significant_Bit(X : in out Big_Unsigned);
  procedure Set_Most_Significant_Bit (X : in out Big_Unsigned);
\end{lstlisting}
The first procedure sets the least significant (the most right) bit of
X to 1. Then, X is after calling this procedure always odd. The second
procedure sets the most significant (the most left) bit of X to
1. This procedure ensures that X has always n bits.

\hhline
\begin{lstlisting}{}
  function Is_Odd (X : Big_Unsigned) return Boolean;
  function Is_Even(X : Big_Unsigned) return Boolean;
\end{lstlisting}
The function \texttt{Is\_Odd()} returns true if X is odd, and the
function \texttt{Is\_Even()} returns true if X is even.

\hhline
\begin{lstlisting}{}
  procedure Inc(X : in out Big_Unsigned);
  procedure Dec(X : in out Big_Unsigned);
\end{lstlisting}
The procedure \texttt{Inc()} increases X by 1, and the procedure
\texttt{Dec()} decreases X by 1.x

\hhline
\begin{lstlisting}{}
  function Shift_Left (Value  : Big_Unsigned;
                       Amount : Natural) return Big_Unsigned;

  function Shift_Right(Value  : Big_Unsigned;
                       Amount : Natural) return Big_Unsigned;
\end{lstlisting}
The function \texttt{Shift\_Left()} calculates
$Value\,*\,2^{Amount}$. And the function \texttt{Shift\_Right()}
calculates $\lfloor Value\, /\,
2^{Amount}\rfloor$.\\

\noindent\textbf{Example:}
\begin{lstlisting}{}
    X : Big_Unsigned := 2#101011#;
  begin
    X := Shift_Right(X, 2);  -- X := 2#001010#
  end;
\end{lstlisting}
X is originally assigned by $2\#101011\#$, decimally equals 43. After
calling the function \texttt{Shift\_Right()}, X is shifted to right
for 2 positions and becomes $2\#001010\#$, decimally equals $\lfloor
43\,/\,2^2 \rfloor$ = 10.

\hhline
\begin{lstlisting}{}
  function Rotate_Left (Value  : Big_Unsigned;
  							   Amount : Natural) return Big_Unsigned;
  function Rotate_Right(Value  : Big_Unsigned; 
  							   Amount : Natural) return Big_Unsigned;
\end{lstlisting}
The function \texttt{Rotate\_Left()} makes rotation of a value to the
left. It mathematically calculates $((Value * 2^{Amount})\; \oplus\;
(\lfloor Value\, /\, 2^{Size - Amount}\rfloor))$ $\bmod\; 2^{Size}$.
The parameter \texttt{Amount} will be calculated through:

\begin{lstlisting}{}
  L : constant Natural := Amount mod Size;
\end{lstlisting}
then, rotation is made by combining function \texttt{Shift\_Left()}
and \texttt{Shift\_Right()}:

\begin{lstlisting}{}
  return Shift_Left(Value,L) xor Shift_Right(Value, Size-L);
\end{lstlisting}
The function \texttt{Rotate\_Right()} makes rotation of a value to the
right. It mathematically calculates $((Value * 2^{Size - Amount})\;
\oplus\; (\lfloor Value / 2^{Amount}\rfloor))$ $\bmod\; 2^{Size}$, as
shown in the following:
\begin{lstlisting}{}
  R : constant Natural := Amount mod Size;
  return Shift_Right(Value,R) xor Shift_Left(Value, Size-R);
\end{lstlisting}

\hhline
\begin{lstlisting}{}
  function Get_Random return Big_Unsigned;
\end{lstlisting}
This function generates a random number in the interval [
  $0\ldots$\texttt{Big\_Unsigned\_Last} ].

\hhline
\begin{lstlisting}{}
  function Lowest_Set_Bit(X : Big_Unsigned) return Natural;
\end{lstlisting}
This function returns the position of the least significant bit in X,
which has the value "1".\\

\noindent\textbf{Exception:} If X = 0
(\texttt{Big\_Unsigned\_Zero}):\quad \texttt{Is\_Zero\_Error}.

\hhline
\begin{lstlisting}{}
  function Gcd(Left, Right : Big_Unsigned) return Big_Unsigned;
\end{lstlisting}
The function calculates the greatest common divisor of the two values.\\

\hhline
\begin{lstlisting}{}
  function Length_In_Bytes(X : Big_Unsigned) return Natural;
  function Bit_Length     (X : Big_Unsigned) return Natural;
\end{lstlisting}
The function \texttt{Length\_In\_Bytes()} calculates the number of
bytes which are required to output X as a byte array. The function
\texttt{Bit\_Length()} returns the bit length of X.

\hhline
\begin{lstlisting}{}
  function To_Big_Unsigned(X : Bytes)        return Big_Unsigned;
  function To_Bytes       (X : Big_Unsigned) return Bytes;
\end{lstlisting}
The two functions provide conversion between values of
\texttt{Big\_Unsigned} and bytes.  The function
\texttt{To\_Big\_Unsigned()} converts X of bytes to a
\texttt{Big\_Unsigned} value B. The function \texttt{To\_Bytes()}
converts a variable X of the type \texttt{Big\_Unsigned} to a byte
array B.\\

\noindent\textbf{Exception:} If the bit length of X is greater than
\texttt{Size}:\quad \texttt{Constraint\_Error}.

\hhline
\begin{lstlisting}{}
  function To_Mod_Types   (X : Big_Unsigned) return Mod_Types;
  function To_Big_Unsigned(X : Mod_Types)    return Big_Unsigned;
\end{lstlisting}
The two functions make conversion between values of
\texttt{Mod\_Types} and \texttt{Big\_Unsigned}.\\

\noindent\textbf{Exception:}
In the function \texttt{To\_Big\_Unsigned()}, if
the bit length of X is greater than \texttt{Size}:\quad
\texttt{Constraint\_Error}.

\hhline
\begin{lstlisting}{}
  function To_String(Item : Big_Unsigned;
                     Base : Number_Base := 10) return String;
  function To_Big_Unsigned(S : String)         return Big_Unsigned;
\end{lstlisting}
The function \texttt{To\_String()} converts \texttt{Item} to a
string. The conversion can be done with a Base (Base $\in \{2,\ldots
,16\}$). The output string has the following structures:
\begin{itemize}
\item \textbf{Base  = 10: } a number with the base 10 (e.g.''1325553'')
\item \textbf{Base /= 10: } a number with the base B\# (e.g.''12\#AB45623A3402\#'')
\end{itemize}
The function \texttt{To\_Big\_Unsigned()} makes transformation of a
string S to a \texttt{Big\_Unsigned} value. The string will be
displayed as a number of decimal notation or with a base (Base $\in
\{2,\ldots ,16\})$ and should have the following structures:
\begin{itemize}
\item a number with base B\# (e.g.''16\#FF340A12B1\#'')
\item a number with base 10 (e.g.''333665'')
\end{itemize}

\noindent\textbf{Exceptions:}
\begin{itemize}
\item If S is a null string:\quad \texttt{Conversion\_Error}.
\item If S has an invalid base:\quad \texttt{Conversion\_Error}.
\item If S has invalid characters:\quad \texttt{Conversion\_Error}.
\end{itemize}

\hhline
\begin{lstlisting}{}
  procedure Put(Item : in Big_Unsigned;
                Base : in Number_Base := 10);

  procedure Put_Line(Item : in Big_Unsigned;
                     Base : in Number_Base := 10);
\end{lstlisting}
The procedure \texttt{Put()} outputs the \texttt{Big\_Unsigned}
variable in standard format with a Base (Base $\in \{2,\ldots
,16\})$. The procedure \texttt{Put\_Line()} outputs the value in
standard format including a line break.

\hhline
\begin{lstlisting}{}
  procedure Big_Div(Dividend, Divisor   : in  Big_Unsigned;
                    Quotient, Remainder : out Big_Unsigned);

  procedure Short_Div(Dividend  : in  Big_Unsigned;
                      Divisor   : in  Mod_Type;
                      Quotient  : out Big_Unsigned;
                      Remainder : out Mod_Type);
\end{lstlisting}
The procedure \texttt{Big\_Div()} calculates the quotient
(\texttt{Quotient}) and the rest (\texttt{Remainder}) of two
\texttt{Big\_Unsigned} variables, meanwhile the procedure
\texttt{Short\_Div()} makes the same calculation between a
\texttt{Big\_Unsigned} variable and a \texttt{Mod\_Type} variable. The
following rules are applied:
\begin{itemize}
\item Quotient := $\lfloor \frac{Dividend}{Divisor} \rfloor$
\item Remainder :=  $Dividend\; \bmod\; Divisor$
\end{itemize}
\textbf{Exception:} If $Divisor$ = 0:\quad \texttt{Division\_By\_Zero}.

%%%%%%%%%%%%%%%%%%%%%%%%%%%%%%%%%%%%%%%%%%%%%%%%%%%%%%%%%%%%%%%%%%%%
%%%%%%%%%%%%%%%%%%%%%%%%%%%%%%%%%%%%%%%%%%%%%%%%%%%%%%%%%%%%%%%%%%%%


\section{Mod\_Utils}
The \texttt{Crypto.Types.Big\_Numbers.Mod\_Utils} is another separate
body of \texttt{Crypto.Types.\-Big\_Numbers}. Functions and
procedures, which are needed for Public-Key-Cryptography, will be
introduced. All the operations in this package are mod N.
\begin{lstlisting}{}
  function Add (Left, Right, N : Big_Unsigned) return Big_Unsigned;
  function Sub (Left, Right, N : Big_Unsigned) return Big_Unsigned;
  function Div (Left, Right, N : Big_Unsigned) return Big_Unsigned;
  function Mult(Left, Right, N : Big_Unsigned) return Big_Unsigned;
\end{lstlisting}
These are the four basic functions for two values, the result should be mod N.\\

\noindent\textbf{Exception in \texttt{Div()}:} If Right = 0
(\texttt{Big\_Unsigned\_Zero}): \texttt{Division\_By\_Zero}.


\hhline
\begin{lstlisting}{}
  function Pow(Base, Exponent, N : Big_Unsigned) return Big_Unsigned;
\end{lstlisting}
It calculates $Base^{Exponent}$ (mod N).

\hhline
\begin{lstlisting}{}
  function Get_Random (N : Big_Unsigned) return Big_Unsigned;
\end{lstlisting}
It returns a random \texttt{Big\_Unsigned} variable (mod N).

\hhline
\begin{lstlisting}{}
  function Inverse (X,N: Big_Unsigned) return Big_Unsigned;
\end{lstlisting}
It returns the inverse (with respect of multiplication) of X (mod
N). If no inverse of X exists, \texttt{Big\_Unsigned\_Zero} will be
returned.

\hhline
\begin{lstlisting}{}
  function Get_Prime  (N : Big_Unsigned) return Big_Unsigned;
  function Get_N_Bit_Prime(N : Positive) return Big_Unsigned;
\end{lstlisting}
The function \texttt{Get\_Prime()} calculates with an overwhelming
probability a prime variable P (mod N), while the function
\texttt{Get\_N\_Bit\_Prime()} calculates a N-bit prime result.\\

\noindent\textbf{Exception:}
\begin{itemize}
\item In the function \texttt{Get\_Prime()}, if N $<=$ 2:\quad\texttt{Constraint\_Error}.
\item In the function \texttt{Get\_N\_Bit\_Prime()}, if N = 1 or\; N $>$ \texttt{Size}:\quad\texttt{Constraint\_Error}.
\end{itemize}

\hhline
\begin{lstlisting}{}
  function Is_Prime(X : Big_Unsigned) return Boolean;
\end{lstlisting}
This function returns true if X is prime or false if X is not prime
with an overwhelming probability. An internal function
\texttt{Pass\_Prime\_Test()} is called to check X.\\

\paragraph{How it  works.}
\begin{enumerate}
\item Test if a one digit prime (2,3,5,7) divides $X$.
\item Test if a two digit prime number divides $X$.
\item Test if a three digit prime number divides $X$.
\item Test if X is a "Lucas-Lehmer" prime\footnote{http://en.wikipedia.org/wiki/Lucas-Lehmer\_primality\_test, November 2012.}.
\item Compute N random \texttt{Big\_Unsigned} values and test if one
  of those is an Miller-Rabin
  wittness\footnote{http://en.wikipedia.org/wiki/Miller-Rabin\_primality\_test,
    November 2012.} ($\,1<N<51$, where N depends on the bit length of X).
\end{enumerate}

\hhline
\begin{lstlisting}{}
  function Looks_Like_A_Prime(X : Big_Unsigned) return Boolean;
\end{lstlisting}
It's a weaker but faster prime test than function
\texttt{Is\_Prime()}. It returns false with high probability if $X$ is
not prime. An internal function \texttt{Pass\_Prime\_Test()} is called
to check X.

\paragraph{How it works.} This function acts much the same as the
function \texttt{Is\_Prime()}, with the difference that the
Miller-Rabin-Test is replaced by a simpler but unreliable prime number
test, which runs 2-50 times to generate random numbers and test if it
is a divisor of X (while Miller-Rabin-Test tests if it is a
Miller-Rabin wittness). When the divisor of X doesn't exist, the
function returns true.

\hhline
\begin{lstlisting}{}
  function Passed_Miller_Rabin_Test(X : Big_Unsigned;
                                   S : Positive) return Boolean;
\end{lstlisting}
It returns true only when X passes a certain number of Miller-Rabin
tests (S iterations). The probability that a pseudo prime number
passes this test is smaller than $\frac{1}{2^{2S}}$.

\hhline
\begin{lstlisting}{}
  function Jacobi(X, N : Big_Unsigned) return Integer;
\end{lstlisting}
This function computes the Jacobi-symbol of X from $Z_N$.\\

\noindent\textbf{Precondition:} N must be odd.\\

\noindent\textbf{Returned values:} \\
\begin{tabular}{p{\textwidth}}
\quad 0 : if X mod N = 0\\
\quad 1 : if X is a quadratic resuide mod N (for some integer b, $X=b^2$ mod N)\\
\quad -1 : if X is a quadratic non-resuide mod N (there is no such b)
\end{tabular}\ \\

\noindent\textbf{Exception:} If X is even:\quad \texttt{Constraint\_Error}.


%%%%%%%%%%%%%%%%%%%%%%%%%%%%%%%%%%%%%%%%%%%%%%%%%%%%%%%%%%%%%%%%
%%%%%%%%%%%%%%%%%%%%%%%%%%%%%%%%%%%%%%%%%%%%%%%%%%%%%%%%%%%%%%%%
\section{Binfield\_Utils}
It's also a separate body of
\texttt{Crypto.Types.Big\_Numbers}. Functions and procedures in this
package receive variables of type \texttt{Big\_Unsigned} from the
field $GF(2^m)$ as parameters. This package is specified in
\texttt{Crypto.Types.Big\_Numbers.Binfield\_Utils}.\\

\noindent\textbf{Example:}
\begin{lstlisting}{}
  -- The 1st, 2nd and 6th bit of A are to be set.
  -- After calculation A is equal to z^5 + z + 1
  A := Big_Unsigned_One;
  A := A xor Shift_left(Big_Unsigned_One,1);
  A := A xor Shift_left(Big_Unsigned_One,5);
\end{lstlisting}
In the following functions, the variable F of type
\texttt{Big\_Unsigned} is an irreducible polynom f(z) of a degree
$m$.

\begin{lstlisting}{}
  function B_Add(Left, Right :Big_Unsigned) return Big_Unsigned;
  function B_Sub(Left, Right :Big_Unsigned) return Big_Unsigned;
\end{lstlisting}
The two functions calculate \texttt{Left} xor \texttt{Right}, which
corresponds to addition or subtraction in $GF(2^m)$.


\hhline
\begin{lstlisting}{}
  function B_Mult(Left, Right, F :Big_Unsigned) return Big_Unsigned;
  function B_Div (Left, Right, F :Big_Unsigned) return Big_Unsigned;
\end{lstlisting}
The function \texttt{B\_Mult()} computes $Left*Right$ mod F. In the
function \texttt{B\_Div()} it uses \texttt{B\_Mult()} with the inverse
of the value \texttt{Right}.

\hhline
\begin{lstlisting}{}
  function B_Square(A, F : Big_Unsigned) return Big_Unsigned;
\end{lstlisting}
It computes $A^2$ mod F.

\hhline
\begin{lstlisting}{}
  function B_Mod(Left, Right : Big_Unsigned) return Big_Unsigned;
\end{lstlisting}
It computes \texttt{Left} mod \texttt{Right}.

\hhline
\begin{lstlisting}{}
  function B_Inverse(X, F : Big_Unsigned) return Big_Unsigned;
\end{lstlisting}
It returns $X^{(-1)}$ mod F.
