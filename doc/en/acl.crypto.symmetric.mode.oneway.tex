\chapter{Crypto.Symmetric.Mode.Oneway}
This generic package works with one-way blockciphers in one-way modes. It integrates a one-way block cipher with a feedback and also some easy operations ($+,xor$). A one-way mode will be initialized with a random start value (Initial Value (IV)). The ciphertext is therefore not only dependent on the used mode, plaintext and key, but also on the random start value. When users encrypt a plaintext twice with the same key in the same mode but different IVs, then they will get different ciphertexts, i.e. a oneway mode encrypts two plaintext blocks $P_1$ and $P_2$ with $P_1=P_2$ to two ciphertext blocks $C_1$ and $C_2$ with overwhelming probability that $C_1\neq C_2$. So that it is now possible to encrypt more messages with the same key.\\
\textbf{Notice: To decrypt a ciphertext the key and start value by encryption are required.} For this reason the start value should be always kept together with the related ciphertext. \textbf{The security of a mode is independent from the familarity of the start value.} Hense, man multiplies usually the start value with the ciphertext as the final output by attaching the ciphertext to the start value ($C'=IV||C$).
%%%%%%%%%%%%%%%%%%%%%%%%%%%%%%%%%%%%%%%%%%%%%%%%%%%%%%%%%%%%%%
%%%%%%%%%%%%%%%%%%%%%%%%%%%%%%%%%%%%%%%%%%%%%%%%%%%%%%%%%%%%%%
\subsubsection*{Remarks}
\begin{itemize}
\item In a oneway mode it goes similarly as in a normal mode. If a normal mode can be used also as a oneway mode, then you should still prefer oneway mode, because it's nimbler.  
\item The API of this package is the same as in normal modes (\ref{API-Mode}).
\item Supported oneway modes:
\begin{itemize}
\item Cipher Feedback Mode (CFB) (\ref{CipherFeedbackMode})
\item Counter Mode (CTR) (\ref{CounterMode})
\item Output Feedback Mode (OFB) (\ref{OutputFeedbackMode})
\end{itemize}
\end{itemize}
\subsubsection*{Example}
\begin{lstlisting}{}
  with Crypto.Types;
  with Ada.Text_IO;
  with Crypto.Symmetric.Oneway_Blockcipher_Twofish128;
  with Crypto.Symmetric.Mode.Oneway_CTR;
  procedure Example_Mode_Oneway is
	 use Ada.Text_IO; use Crypto.Types;
    package TF128 renames 
                  Crypto.Symmetric.Oneway_Blockcipher_Twofish128;
    package Twofish128 is new
                  Crypto.Symmetric.Mode.Oneway_CTR(TF128);
 	 use Twofish128;
    Key: B_Block128 := (16#2b#, 16#7e#, 16#15#, 16#16#, 16#28#,
                        16#ae#, 16#d2#, 16#a6#, 16#ab#, 16#f7#,
                        16#15#, 16#88#, 16#09#, 16#cf#, 16#4f#,
                        16#3c#);
    IV: B_Block128 := (15 => 1, others => 0);
     --Plaintext
    P_String: String :="All your base are belong to us! ";
     --Plaintext will be divided in 2*64 bit blocks.
    P: array(1..2) of B_Block128 := 
			(To_B_Block128(To_Bytes(P_String(1..16))),
			 To_B_Block128(To_Bytes(P_String(17..32))));
    C: array(0..2) of B_Block128;      --Ciphertext
  begin
    Init(Key, IV);         --Initialization
    C(0):= IV;             --1. Ciphertext block = start value
    for I in P'Range loop
	    Encrypt(P(I),C(I));   --Encryption
    end loop;
     --For decryption the start value will be reinitialized.
    Set_IV(C(0));
    for I in P'Range loop
	    Decrypt(C(I),P(I));     --Decryption
       Put(To_String(To_Bytes(P(I))));
    end loop;
  end Example_Mode_Oneway;
\end{lstlisting}