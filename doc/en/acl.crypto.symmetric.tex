 \chapter{Crypto.Symmetric}

\subsubsection{Description}
This is the root package for symmetric cryptography.\\

\subsubsection{Function}
This package provides the symmetric part of the ACL with direct access to
Crypto.Types, which contains the fundamental types and corresponding base
functions.
 
\section{Logical Setup}
A block cipher is separated into three logical layers. An application programmer
should solely use the API of the upmost layer. Only with this API, a secure
application of a block cipher can be guaranteed.

\subsubsection{Layer 1: Algorithm}
This layer provides the pure algorithm of a symmetric cipher. The API of
the different algorithms is identical and provides as an interface for the
block cipher layer.
With each algorithm, a reference to the corresponding source (e.g. a paper)
is provided. As a result, it is possible to verify the implementation
by consulting this reference. Generally, this is always a good thing to do.
Additional information regarding the particular algorithms and ciphers can be
found within the next chapter. \\ \ \\
\textbf{Comment:}\\
The API of this layer should only be used for generating a block cipher.
It should by no means be used for any other purpose. There is no scenario
where this would make any sense. This API only serves as an interface for
layer 2.

\subsubsection{Layer 2: Block Cypher}
At this layer, a block cipher is generated from the algorithm in the
insecure ECB-mode (Electronic Code Book). You should only use this API when
you are familiar with symmetric block ciphers. Actually, this API serves as
an interface for layer 3.
Chapter \ref{block} explains how a block cipher is
generated from an algorithm.

\subsubsection{Layer 3: Mode}
This layer transforms a block cipher into a reasonable mode. There are different
modes for different use cases, each of them having there particular assets and
drawbacks. Chapter \ref{modus} provides a detailed description.

\begin{figure}
  \begin{center}
    \huge
    \begin{tabular}{|c @{\ } c|}\hline
      III. & Mode\\
      \hline
      II. & Block Cypher\\
      \hline
      I. & Algorithm\\
    \hline
    \end{tabular}
  \end{center}
\caption{The three-layered architecture of a symmetric (block) cipher}
\end{figure}

\subsubsection{Comment}
By using this three-layered architecture, it is also possible to implement
stream ciphers. In order to do this, you have to provide the first layer with
a procedure for receiving the next n bits of the key stream. Afterwards, you
can construct an oneway block cipher (Chapter \ref{oneblock}) and and transform
it to the oneway-OFB or oneway-counter mode. The ACL currently does not provide
any stream cipher. However, there are plans for implementing the stream cipher
Helix \cite{helix}.
