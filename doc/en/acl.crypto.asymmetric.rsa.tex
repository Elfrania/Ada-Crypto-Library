\chapter{Crypto.Asymmetric.RSA}
RSA is an algorithm for public-key cryptography that is based on the
presumed difficulty of factoring large integers. RSA stands for Ron
Rivest, Adi Shamir and Leonard Adleman, who first publicly described
it in \cite{PKCS} in 1978. Plaintext(-blocks) or ciphertext(-blocks)
are encrypted or decrypted in OAEP (Optimal Asymmetric Encryption
Padding), which is recommended in PKCS1-v2-1 (Public-Key Cryptography
Standards) \cite{PKCS}.
\subsubsection*{OEAP Details}
\begin{itemize}
\item The implementation uses SHA1 inside MGF (Mask Generation
  Function: a function generating an arbitrary number of bits for a
  given input)
\item The implementation doesn't support the optional label L, i.e. L
  is always an empty string.
\end{itemize}
\subsubsection*{Generic Part}
\begin{lstlisting}{}
  generic
    Size : Positive;
\end{lstlisting}
\textbf{Exception:} If Size $< 512$ :\quad \texttt{Constraint\_Size\_Error}.\\
\section{API}
\subsection*{Types}
\begin{lstlisting}{}
  subtype RSA_Number is Bytes(0..Size/8-1);
  type Public_Key_RSA is private;
  type Private_Key_RSA is private;
\end{lstlisting}
The \texttt{RSA\_Number} is a byte array which interprets a
number. Its first element ($'First$) corresponds the most significant
byte and the last element ($'Last$) corresponds the least significant
byte of the array.\\
\begin{lstlisting}{}
  type Public_Key_RSA is record
    N : Big_Unsigned; -- Modulus
    E : Big_Unsigned; -- Public exponent
  end record;
  type Private_Key_RSA is record
    N    : Big_Unsigned; -- Modulus = p * q
    D    : Big_Unsigned; -- private exponent = (e ^ -1) mod Phi
    P    : Big_Unsigned; -- prime p
    Q    : Big_Unsigned; -- prime q
    Phi  : Big_Unsigned; -- = (p - 1) * (q - 1)

    --  Chinese Remainder Theory (CRT) forms of the
    --  private exponent
    DP   : Big_Unsigned; -- = d mod (p - 1)
    DQ   : Big_Unsigned; -- = d mod (q - 1)
    QInv : Big_Unsigned; -- = (q ^ -1) mod p
  end record;
\end{lstlisting}
The term \texttt{Public\_Key\_RSA} has two components, and
\texttt{Private\_Key\_RSA} has eight components. They are used in
internal functions to generate key pairs.\\
\subsubsection*{High-Level-API}
\begin{lstlisting}{}
  procedure Gen_Key(Public_Key  : out Public_Key_RSA;
                    Private_Key : out Private_Key_RSA);
\end{lstlisting}
This procedure generates randomly a pair of keys, which are a public
key (\texttt{Public\_Key}) and a private key (\texttt{Private\_Key}),
by calculating all components of the key pair.\\
\begin{lstlisting}{}
  function Verify_Key_Pair(Private_Key : Private_Key_RSA;
                           Public_Key  : Public_Key_RSA)
                           return Boolean;
\end{lstlisting}
This function returns true if the two keys, \texttt{Private\_Key} and
\texttt{Public\_Key}, are a pair, if not, then false is
returned.

\hhline
\begin{lstlisting}{}
  function OAEP_Encrypt(Public_Key : in  Public_Key_RSA;
                        Plaintext  : in  Bytes)
                        return RSA_Number;
\end{lstlisting}
The function \texttt{OAEP\_Encrypt()} encrypts a plaintext(-block)
(\texttt{Plaintext}) in the OEAP-Process and returns a ciphertext.
The plaintext $M$ is encoded as a part of a data block $DB$, where the
$lHash$ is a 160-bit hash value of an empty string. The data block
$DB$ of length $k$ can be formed as:
\begin{equation*}
DB=lHash||PS||0\mbox{x}01||M\,,
\end{equation*}
where the string $PS$ consists of $k-160-8-|M|$ zero bits, its length
may be zero, and 0x01 is a hexadecimal value. The data block is
encoded (to a message $EM$) and encrypted to the ciphertext.  Detailed
information about the \texttt{OAEP\_Encrypt()} can be found in
\cite{PKCS}. The length limitation of the plaintext is defined as:
\texttt{Size/8}$-42\;(42=2*20+2)$.\\

\noindent\textbf{Exception:}\\
If the length of the plaintext is greater than the limitation:\quad
\texttt{Plaintext\_Too\_Long\_Error};

\hhline
\begin{lstlisting}{}
  function OAEP_Decrypt(Private_Key : in  Private_Key_RSA;
                        Ciphertext  : in  RSA_Number)
                        return Bytes;
\end{lstlisting}
This function decrypts a ciphertext(-block) (\texttt{Ciphertext})
under the private key. If the used private key and its related public
key are a pair generated by \texttt{Gen\_Key()}, then the ciphertext
is recovered to a data block as the structure shown in
\texttt{OAEP\_Encrypt()}, and the plaintext is separated. Detailed
information about the \texttt{OAEP\_Decrypt()} can be found in
\cite{PKCS}.\\

\noindent\textbf{Exception:}\\
If one of the following conditions is
met, then a \texttt{Decrypt\_Error} is raised.
\begin{enumerate}
\item the ciphertext is greater
 than the RSA-Modulus $N$\,,
\item the first value of the $EM$
 (\texttt{OAEP\_Encrypt()}) is not zero\,,
\item the term $lHash'$ of the new data block is not equal $lHash$ in
  \texttt{OAEP\_Encrypt()}.
\end{enumerate}

\hhline
\begin{lstlisting}{}
  procedure Get_Public_Key(Public_Key : in Public_Key_RSA;
                           N          : out RSA_Number;
                           E          : out RSA_Number);
\end{lstlisting}
The procedure decomposes a public key (\texttt{Public\_Key}) into two
components:
\begin{itemize}
\item One size-bit RSA modulus $N$ where $N=PQ$, where $P,Q$ are prime numbers
\item One public RSA exponent $E$
\end{itemize}
The public key can be reconstructed later with those two components.

\hhline
\begin{lstlisting}{}
  procedure Get_Private_Key (Private_Key : in Private_Key_RSA;
                             N           : out RSA_Number;
                             D           : out RSA_Number;
                             P           : out RSA_Number;
                             Q           : out RSA_Number;
                             Phi         : out RSA_Number;
                             DP          : out RSA_Number;
                             DQ          : out RSA_Number;
                             QInv        : out RSA_Number;
\end{lstlisting}
The procedure decomposes a private key (\texttt{Private\_Key}) into
the following components:
\begin{itemize}
\item One size-bit RSA modulus $N$ where $N=PQ$, $P,Q$ are prime numbers
\item One private RSA exponent $D$
\item Primes $P$ and $Q$
\item $\phi(N)=(P-1)(Q-1)$
\item Chinese Remainder Theory (CRT) forms of the private exponent $DP$, $DQ$, and $QInv$
\end{itemize}
The private key can be reconstructed later with those components.

\hhline
\begin{lstlisting}{}
  procedure Set_Public_Key(N          : in RSA_Number;
                           E          : in RSA_Number;
                           Public_Key : out Public_Key_RSA);
\end{lstlisting}
During the procedure a public key \texttt{Public\_Key} can be
(re-)constructed. The following values are needed:
\begin{itemize}
\item One size-bit RSA modulus $N$
\item One public RSA exponent $E$
\end{itemize}
\textbf{Exception:}\\
 If the public key is invalid, where the length
of $N$ is not equal the \texttt{Size}, or the term $E$ is even or
smaller than 3 :\quad \texttt{Constraint\_Error}.


\hhline
\begin{lstlisting}{}
  procedure Set_Private_Key(N           : in RSA_Number;
                            D           : in RSA_Number;
                            P           : in RSA_Number;
                            Q           : in RSA_Number;
                            Phi         : in RSA_Number;
                            Private_Key : out Private_Key_RSA);
  procedure Set_Private_Key(N           : in Big_Unsigned;
                            D           : in Big_Unsigned;
                            P           : in Big_Unsigned;
                            Q           : in Big_Unsigned;
                            Phi         : in Big_Unsigned;
                            Private_Key : out Private_Key_RSA);
\end{lstlisting}
The two procedures can both be used to (re-)construct a private
key. The following values are required as parameters:
\begin{itemize}
\item One size-bit RSA modulus $N$ where $N=PQ$,
\item Prime numbers $P$ and $Q$
\item One private RSA exponent $D$
\item $\phi(N)=(P-1)(Q-1)$
\end{itemize}
The CRT forms of the private exponent are computed from $P$, $Q$, and $D$.

\noindent\textbf{Exception:}\\ If one of the following conditions about the
private key is met, then a \texttt{Constraint\_Error} is raised:
\begin{itemize}
\item the length of $N$ is not equal the \texttt{Size}\,,
\item $P$ less than or equal to $Q$
\item the term $D$ is even, or its bit value is smaller than or equal 2\,,
\item the term $\phi(N)$ is odd, or its length is smaller than
  \texttt{Size}-2\,,
\item the greatest common divisor of $D$ and $\phi(N)$ is not 1\,.
\item RCC notes this GCD rule does not match the code.  Bug?
\end{itemize}

\subsubsection*{Low-Level-API}
The Low-Level-API can be used only when users know exactly what it
does. Naive usage of the API can lead to critic security problems,
cause identical plaintexts can be encrypted to identical ciphertexts.
\begin{lstlisting}{}
  procedure Encrypt(Public_Key : in  Public_Key_RSA;
                    Plaintext  : in  RSA_Number;
                    Ciphertext : out RSA_Number);
  procedure Encrypt(Public_Key : in  Public_Key_RSA;
                    Plaintext  : in  Big_Unsigned;
                    Ciphertext : out Big_Unsigned);
\end{lstlisting}
The procedures encrypt a plaintext to a ciphertext with a public
key. They use the "naive" RSA-Process ($C=P^{K.E}$(mod $K.N$)).\\

\noindent\textbf{Exception:}\\
If the public key is invalid, where the length of $N$ is not equal the
\texttt{Size}, or the term $E$ is even or smaller than 3 :\quad
\texttt{Invalid\_Public\_Key\_Error}.

\hhline
\begin{lstlisting}{}
  procedure Decrypt(Private_Key : in  Private_Key_RSA;
                    Ciphertext  : in  RSA_Number;
                    Plaintext   : out RSA_Number);
  procedure Decrypt(Private_Key : in  Private_Key_RSA;
                    Ciphertext  : in  Big_Unsigned;
                    Plaintext   : out Big_Unsigned);
\end{lstlisting}
The two procedures can be both used to decrypt a ciphertext. It checks
at first if the private key is valid or not, and then it decrypts the
ciphertext to a plaintext with the private key. They use the "naive"
RSA-Process ($P=C^{K.D}$ (mod $K.N$)).\\

\noindent\textbf{Exception:}\\ If one of the following conditions
about the private key is met, then a \texttt{Decrypt\_Error} is
raised:
\begin{enumerate}
\item the length of $N$ is not equal the \texttt{Size}\,,
\item the term $D$ is even, or its bit value is smaller than or equal 2\,,
\item the term $\phi(N)$ is odd, or its length is smaller than \texttt{Size}-2\,,
\item the greatest common divisor of $D$ and $\phi(N)$ is not 1\,.
\end{enumerate}

\section{Example}
\begin{lstlisting}{}
  with Crypto.Types; use Crypto.Types;
  with Crypto.Asymmetric.RSA;
  with Ada.Text_IO; use Ada.Text_IO;
  procedure Example_RSA is
    package RSA is new Crypto.Asymmetric.RSA(512);
    use RSA;
    Message : Bytes := To_Bytes("Good Luck!");
    Public_Key : Public_Key_RSA;
    Private_Key : Private_Key_RSA;
  begin
    Gen_Key(Public_Key, Private_Key); -- Generation of key pair
    declare
      -- Encryption & Decryption
	  Ciphertext:RSA_Number:= OAEP_Encrypt(Public_Key, Message);
	  Plaintext:Bytes:= OAEP_Decrypt(Private_Key, Ciphertext);
    begin
      Put(To_String(Ciphertext)); -- Output of ciphertext
	   New_Line;
      Put(To_String(Plaintext)); -- Output of plaintext
    end;
  end Example_RSA;
\end{lstlisting}
