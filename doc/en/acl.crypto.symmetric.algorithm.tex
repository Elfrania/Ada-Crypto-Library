\chapter{Crypto.Symmetric.Algorithm}\label{Algorithm}
All child packages of the package \texttt{Crypto.Symmetric.Algorithm}
are symmetric ciphers or hash functions. They have direct access to
individual operations, which use a cipher or a hash function. However
users should use the existing API.

%%%%%%%%%%%%%%%%%%%%%%%%%%%%%%%%%%%%%%%%%%%%%%%%%%%%%%%%%%%%%%%%%%%%%%
%%%%%%%%%%%%%%%%%%%%%%%%%%%%%%%%%%%%%%%%%%%%%%%%%%%%%%%%%%%%%%%%%%%%%%

\section{Symmetric Ciphers}
\subsubsection*{General}
The following procedures are defined and used within symmetric ciphers
of the ACL.
\begin{itemize}
\item One procedure \texttt{Prepare\_Key()}; it generates a cipher key
  from the delivered value.\\ \textbf{Example:}
\begin{lstlisting}{}
  procedure Prepare_Key(Key       : in  B_Block128;
                        Cipherkey : out Cipherkey_Type);
\end{lstlisting}
\item One procedure \texttt{Encrypt()}; it converts the input
  plaintext into a output ciphertext.\\

\noindent\textbf{Example:}
\begin{lstlisting}{}
  procedure Encrypt(Cipherkey  : in  Cipherkey_Type;
                    Plaintext  : in  B_Block128;
                    Ciphertext : out B_Block128);
\end{lstlisting}
\item One procedure \texttt{Decrypt()}; it recovers the input
 ciphertext to associated plaintext.\\

\noindent\textbf{Example:}
\begin{lstlisting}{}
  procedure Decrypt(Cipherkey  : in  Cipherkey_Type;
                    Ciphertext : in  B_Block128;
                    Plaintext  : out B_Block128);
\end{lstlisting}
\end{itemize}

%%%%%%%%%%%%%%%%%%%%%%%%%%%%%%%%%%%%%%%%%%%%%%%%%%%%%%%%%%%%%%%%%%%
%%%%%%%%%%%%%%%%%%%%%%%%%%%%%%%%%%%%%%%%%%%%%%%%%%%%%%%%%%%%%%%%%%%

\subsection{AES}\label{AES}
Advanced Encryption Standard (AES) is one of the best-known ciphers
which was officially standardized by the U.S. National Institute of
Standards and Technology (NIST) in 2001 \cite{AES-FIPS}.  Originally
it was called Rijndael, the name was a combination of the names of the
two inventors, Joan Daemen and Vincent Rijmen. The AES algorithm has a
fixed block size of 128 bits and a key size of 128, 192, or 256 bits.
During the AES algorithm, the input plaintext will be transformed
through a round transformation function 10, 12 or 14 times depending
on its key length, and the final output is the ciphertext
\cite{AES-FIPS}. The number of cycles of repetition are as follows:
\begin{itemize}
\item 10 rounds of repetition for 128 bit keys.
\item 12 rounds of repetition for 192 bit keys.
\item 14 rounds of repetition for 256 bit keys.
\end{itemize}
\subsubsection*{API of AES with 128 bit keys:}
\begin{lstlisting}{}
  type Cipherkey_AES128 is private;
  procedure Prepare_key128(Key       : in  B_Block128;
                           Cipherkey : out Cipherkey_AES128);
  procedure Encrypt128(Cipherkey  : in  Cipherkey_AES128;
                       Plaintext  : in  B_Block128;
                       Ciphertext : out B_Block128);
  procedure Decrypt128(Cipherkey  : in  Cipherkey_AES128;
                       Ciphertext : in  B_Block128;
                       Plaintext  : out B_Block128);
\end{lstlisting}
\subsubsection*{API of AES with 192 bit keys:}
\begin{lstlisting}{}
  type Cipherkey_AES192 is private;
  procedure Prepare_key192(Key       : in  B_Block192;
                           Cipherkey : out Cipherkey_AES192);
  procedure Encrypt192(Cipherkey  : in  Cipherkey_AES192;
                       Plaintext  : in  B_Block128;
                       Ciphertext : out B_Block128);
  procedure Decrypt192(Cipherkey  : in  Cipherkey_AES192;
                       Ciphertext : in  B_Block128;
                       Plaintext  : out B_Block128);
\end{lstlisting}

\subsubsection*{API of AES with 256 bit keys:}
\begin{lstlisting}{}
  type Cipherkey_AES256 is private;
  procedure Prepare_key256(Key       : in  B_Block256;
                           Cipherkey : out Cipherkey_AES256);
  procedure Encrypt256(Cipherkey  : in  Cipherkey_AES256;
                       Plaintext  : in  B_Block128;
                       Ciphertext : out B_Block128);
  procedure Decrypt256(Cipherkey  : in  Cipherkey_AES256;
                       Ciphertext : in  B_Block128;
                       Plaintext  : out B_Block128);
\end{lstlisting}

%%%%%%%%%%%%%%%%%%%%%%%%%%%%%%%%%%%%%%%%%%%%%%%%%%%%%%%%%%%%%%%%%%
%%%%%%%%%%%%%%%%%%%%%%%%%%%%%%%%%%%%%%%%%%%%%%%%%%%%%%%%%%%%%%%%%%

\subsection{Blowfish}
Blowfish was designed in 1993 by Bruce Schneier. It uses a key of
variable length (here the length 128-bit is used) and a 64-bit block
cipher. Blowfish has a key-expansion part and a part of 16-round
Feistel network for data encryption \cite{DBLP:conf/fse/Schneier93}.
\subsubsection*{API of Blowfish:}
\begin{lstlisting}{}
  type Cipherkey_Blowfish128 is private;
  subtype B_Block is Bytes;
  procedure Prepare_Key128(Key      :in  B_Block128;
                          Cipherkey :out Cipherkey_Blowfish128);
  procedure Encrypt128 (Cipherkey  : in  Cipherkey_Blowfish128;
                        Plaintext  : in  B_Block64;
                        Ciphertext : out B_Block64);
  procedure Decrypt128 (Cipherkey  : in  Cipherkey_Blowfish128;
                        Ciphertext : in  B_Block64;
                        Plaintext  : out B_Block64);
\end{lstlisting}

%%%%%%%%%%%%%%%%%%%%%%%%%%%%%%%%%%%%%%%%%%%%%%%%%%%%%%%%%%%%%%%%%%
%%%%%%%%%%%%%%%%%%%%%%%%%%%%%%%%%%%%%%%%%%%%%%%%%%%%%%%%%%%%%%%%%%

\subsection{Serpent}
Serpent is designed for 128-bit block ciphers, and was one of the five
finalists in the Advanced Encryption Standard (AES) contest. If you
put a high value on security and not such a high value on performance,
then you can try this block cipher. The implementation is based on the
Ada-Reference-Implementation by Markus G. Kuhn. Temporarily there is
only API for 256-bit cipher key. API for 128-bit and 192-bit keys is
in planning.
\subsubsection*{API of Serpent:}
\begin{lstlisting}{}
  type Cipherkey_Serpent256 is private;
  procedure Prepare_Key256(Key       : in  B_Block256;
                           Cipherkey : out Cipherkey_Serpent256);
  procedure Encrypt256(Cipherkey  : in  Cipherkey_Serpent256;
                       Plaintext  : in  B_Block128;
                       Ciphertext : out B_Block128);
  procedure Decrypt256(Cipherkey  : in  Cipherkey_Serpent256;
                       Ciphertext : in  B_Block128;
                       Plaintext  : out B_Block128);
\end{lstlisting}

%%%%%%%%%%%%%%%%%%%%%%%%%%%%%%%%%%%%%%%%%%%%%%%%%%%%%%%%%%%%%%%%%%
%%%%%%%%%%%%%%%%%%%%%%%%%%%%%%%%%%%%%%%%%%%%%%%%%%%%%%%%%%%%%%%%%%
\subsection{Triple DES(3DES, TDES, TDEA)}
Triple DES is the common name for the Triple Data Encryption Algorithm
(TDEA or Triple DEA), which applies the Data Encryption Standard (DES)
algorithm to each data block three times, and a TDES key is made of
three DES keys \cite{DES-FIPS}. The 192-bit key is decomposed before
encryption in three 64-bit keys. Due to that DES is software-relative
slow, TDES is the slowest block cipher. It is assumed that this cipher
can not be broken, because DES is over 20 years studied by the best
cryptoanalysts on its weakness. The idea comes from the book
\texttt{Applied Cryptography} by Burce
Scheier\footnote{http://www.schneier.com/book-applied.html, Dec
  2012.}.
\subsubsection*{API of Triple DES:}
\begin{lstlisting}{}
  type Cipherkey_TDES is private;
  procedure Prepare_Key(Key       : in  B_Block192;
                        Cipherkey : out Cipherkey_TDES);
  procedure Encrypt(Cipherkey  : in  Cipherkey_TDES;
                    Plaintext  : in  B_Block64;
                    Ciphertext : out B_Block64);
  procedure Decrypt(Cipherkey  : in  Cipherkey_TDES;
                    Ciphertext : in  B_Block64;
                    Plaintext  : out B_Block64);
\end{lstlisting}
\newpage
%%%%%%%%%%%%%%%%%%%%%%%%%%%%%%%%%%%%%%%%%%%%%%%%%%%%%%%%%%%%%%%%%
%%%%%%%%%%%%%%%%%%%%%%%%%%%%%%%%%%%%%%%%%%%%%%%%%%%%%%%%%%%%%%%%%
\subsection{Twofish}
Twofish is one of the five finalists of the Advanced Encryption
Standard contest, with a block size of 128 bits and a key of length up
to 256 bits. It uses the 16-round Feistel network with a bijective F
function, a transform, a bitwise rotation and a key schedule
\cite{Twofish}. The implementation is based on the C-Reference
implementation by Niels Ferguson and the twofish
specification\footnote{http://www.schneier.com/twofish.html}. In the
ACL it is implemented with 128, 192 or 256 bit keys.
\subsubsection*{API of Twofish with 128 bit keys:}
\begin{lstlisting}{}
  type Cipherkey_Twofish is private;
  subtype Cipherkey_Twofish128 is Cipherkey_Twofish;
  procedure Prepare_Key128(Key      : in  B_Block128;
                           Cipherkey: out Cipherkey_Twofish128);
  procedure Encrypt128(Cipherkey    : in  Cipherkey_Twofish128;
                       Plaintext    : in  B_Block128;
                       Ciphertext   : out B_Block128);
  procedure Decrypt128(Cipherkey    : in  Cipherkey_Twofish128;
                       Ciphertext   : in  B_Block128;
                       Plaintext    : out B_Block128);
\end{lstlisting}
\subsubsection*{API of Twofish with 192 bit keys:}
\begin{lstlisting}{}
  subtype Cipherkey_Twofish192 is Cipherkey_Twofish;
  procedure Prepare_Key192(Key      : in  B_Block192;
                          Cipherkey : out Cipherkey_Twofish192);
  procedure Encrypt192(Cipherkey    : in  Cipherkey_Twofish192;
                       Plaintext    : in  B_Block128;
                       Ciphertext   : out B_Block128);
  procedure Decrypt192(Cipherkey    : in  Cipherkey_Twofish192;
                       Ciphertext   : in  B_Block128;
                       Plaintext    : out B_Block128);
\end{lstlisting}
\subsubsection*{API of Twofish with 256 bit keys:}
\begin{lstlisting}{}
  subtype Cipherkey_Twofish256 is Cipherkey_Twofish;
  procedure Prepare_Key256(Key      : in  B_Block256;
                           Cipherkey: out Cipherkey_Twofish256);
  procedure Encrypt256(Cipherkey    : in  Cipherkey_Twofish256;
                       Plaintext    : in  B_Block128;
                       Ciphertext   : out B_Block128);
  procedure Decrypt256(Cipherkey    : in  Cipherkey_Twofish256;
                       Ciphertext   : in  B_Block128;
                       Plaintext    : out B_Block128);
\end{lstlisting}

%%%%%%%%%%%%%%%%%%%%%%%%%%%%%%%%%%%%%%%%%%%%%%%%%%%%%%%%%%%%%%%%%
%%%%%%%%%%%%%%%%%%%%%%%%%%%%%%%%%%%%%%%%%%%%%%%%%%%%%%%%%%%%%%%%%

\section{Hash Functions}\label{AlgorithmHash}
The iterative hash functions in the ACL can be called through a
Low-Level-API or High-Level-API.
\subsubsection*{Excursion: Iterative Hash Functions}
An iterative hash function $H$ decomposes a message $M$ into $r$
message blocks $m_i$ ($1\leq i \leq r$) of length $n$. All message
blocks will be hashed sequentially. To generate a hash value $h$ for
the complete message $M$, the intermediate outputs are also inputs for
the next round.  In general there exists: $h_i:= H(m_i,h_{i-1})$. For
$i=1$, $h_1:=H(M_1,h_0)$ will be applied, hereby $h_0$ is an initial
value, and will be initialized at the beginning. Messages whose length
is not multiple of $n$ will be inflated by a function
\texttt{Padding()} before encryption until the length is multiple of
$n$.

\subsubsection*{High-Level-API}
The High-Level-API is made of 3 procedures, which calculate hash value
of messages in string, bytes or file names in
string.\\

\noindent\textbf{Example:}
\begin{lstlisting}{}
  procedure Hash  (Message  : in String; Hash_Value: out W_Block160);
  procedure Hash  (Message  : in Bytes;  Hash_Value: out W_Block160);
  procedure F_Hash(Filename : in String; Hash_Value: out W_Block160);
\end{lstlisting}

\subsubsection*{Low-Level-API}
The Low-Level-API of a hash function has:
\begin{itemize}
\item One procedure \texttt{Init()}; it initializes the hash value (\texttt{Hash\_Value}).\\
\textbf{Example:}
\begin{lstlisting}{}
  procedure Init(Hash_Value : out W_Block160);
\end{lstlisting}
\item One procedure \texttt{Round()}; a message block (not the last block) will be hashed.\\
\textbf{Example:}
\begin{lstlisting}{}
  procedure Round(Message_Block : in     W_Block512;
                  Hash_Value    : in out W_Block160);
\end{lstlisting}
\item One function \texttt{Final\_Round()}; the last message block
  (\texttt{Last\_Message\_Block}) will be inflated and the final hash
  value is returned. Because of the \texttt{Padding()} function, the
  length of the last message block in byte (Last\_Message\_Length)
  should be passed as a parameter.\\

\noindent\textbf{Example:}
\begin{lstlisting}{}
  function Final_Round
           (Last_Message_Block  : W_Block512;
            Last_Message_Length : Message_Block_Length512;
            Hash_Value          : W_Block160) return W_Block160;
\end{lstlisting}
\end{itemize}

\subsection{SHA-1}\label{SHA-1}
SHA-1 is a cryptographic hash function for 160-bit message blocks. SHA
stands for "Secure Hash Algorithm". SHA-1 was one of the most widely
used hash functions. In 2005, security flaws were identified in SHA-1
by X. Wang, Y.L. Yin und H.Yu, namely that a mathematical weakness
might exist \cite{DBLP:conf/crypto/WangYY05a}. Due to this new
discovery, it is now not advisable to use this hash function. The only
reason why the ACL supports this kind of hash function is its
tremendous degree of process in standard protocols such as digital
signature standard. But this hash function is expected to be no more
supported by the ACL, so you'd better use other hash functions if
possible. \\ \textbf{Remark:}\\ The performance of SHA-1 is slower
than SHA-256.
\subsubsection*{API}
\begin{lstlisting}{}
  -- high level API
  procedure Hash(Message : in String ; Hash_Value : out W_Block160);
  procedure Hash(Message : in Bytes  ; Hash_Value : out W_Block160);
  procedure F_Hash(Filename: in String; Hash_Value : out W_Block160);
  -- low level API
  procedure Init(Hash_Value     : out    W_Block160);
  procedure Round(Message_Block : in     W_Block512;
                  Hash_Value    : in out W_Block160);
  function Final_Round
           (Last_Message_Block  : W_Block512;
            Last_Message_Length : Message_Block_Length512;
            Hash_Value          : W_Block160) return W_Block160;
\end{lstlisting}
\textbf{Exceptions:}
\begin{itemize}
\item If the message length is greater than $2^{64}-1024$:\quad
  \texttt{SHA1\_Constraint\_Error}.
\item In the procedure \texttt{F\_Hash()}, if the file can not be
  opened:\quad\texttt{File\_Open\_Error}.
 \item In the procedure \texttt{F\_Hash()}, if the size of the read
   data is smaller than 0:\quad\texttt{File\_Read\_Error}.
\end{itemize}

%%%%%%%%%%%%%%%%%%%%%%%%%%%%%%%%%%%%%%%%%%%%%%%%%%%%%%%%%%%%%
\subsection{SHA-256}
SHA-256 is a cryptographic hash function with a 256-bit output (hash
value), and it is a member of SHA-2 family, which is based on
SHA-1. This hash family has a hash value in sufficient length and
makes significant changes in security compared to the SHA-1 hash
function. Since this hash family works with 8*32-bit words, its use in
32-bit processors worths a consideration. If possible, SHA-1 should be
replaced by this new generation of SHA-2 hash family.
\subsubsection*{API}
\begin{lstlisting}{}
  -- high level API
  procedure Hash(Message : in Bytes ; Hash_Value : out W_Block256);
  procedure Hash(Message : in String; Hash_Value : out W_Block256);
  procedure F_Hash(Filename: in String; Hash_Value : out W_Block256);
  -- low level API
  procedure Init(Hash_Value     : out    W_Block256);
  procedure Round(Message_Block : in     W_Block512;
                  Hash_Value    : in out W_Block256);
  function Final_Round
           (Last_Message_Block : W_Block512;
            Last_Message_Length: Message_Block_Length512;
            Hash_Value         : W_Block256) return W_Block256;
\end{lstlisting}
\textbf{Exceptions:}
\begin{itemize}
\item If the message length is greater than $2^{64}-1024$:\quad
  \texttt{SHA256\_Constraint\_Error}.
\item In the procedure \texttt{F\_Hash()}, if the file can not be
  opened:\quad\texttt{File\_Open\_Error}.
\item In the procedure \texttt{F\_Hash()}, if the size of the read
  data is smaller than 0:\quad\texttt{File\_Read\_Error}.
\end{itemize}

%%%%%%%%%%%%%%%%%%%%%%%%%%%%%%%%%%%%%%%%%%%%%%%%%%%%%%%%%%%%%

\subsection{SHA-512}
SHA-512 is the safest member in SHA-2 hash family, which is defined in
Secure Hash
Standard\footnote{http://www.itl.nist.gov/fipspubs/fip180-1.htm}. It
works with 8*64-bit dwords and is therefore particularly suitable for
64-bit architectures. If you are searching for a very safe and
standardized hash function, then this hash function is probably the
one.

%%%%%%%%%%%%%%%%%%%%%%%%%%%%%%%%%%%%%%%%%%%%%%%%%%%%%%%%%%%%%

\subsubsection*{API}
\begin{lstlisting}{}
  -- high level API
  procedure Hash(Message : in Bytes;  Hash_Value : out DW_Block512);
  procedure Hash(Message : in String; Hash_Value : out DW_Block512);
  procedure F_Hash(Filename: in String; Hash_Value: out DW_Block512); 
  -- low level API
  procedure Init(Hash_Value     : out    DW_Block512);
  procedure Round(Message_Block : in     DW_Block1024;
                  Hash_Value    : in out DW_Block512);
  function Final_Round
           (Last_Message_Block  : DW_Block1024;
            Last_Message_Length : Message_Block_Length1024;
            Hash_Value          : DW_Block512) return DW_Block512;
\end{lstlisting}
\textbf{Exceptions:}
\begin{itemize}
\item If the message length is greater than
  $2^{128}-1024$:\quad\texttt{SHA2\_Constraint\_Error}.
\item In the procedure \texttt{F\_Hash()}, if the file can not be
  opened:\quad\texttt{File\_Open\_Error}.
\item In the procedure \texttt{F\_Hash()}, if the size of the read
  data is smaller than 0:\quad\texttt{File\_Read\_Error}.
\end{itemize}

%%%%%%%%%%%%%%%%%%%%%%%%%%%%%%%%%%%%%%%%%%%%%%%%%%%%%%%%%%%%%%%%

\subsection{SHA-384}
The SHA-384 hash function looks like SHA-512. The difference between
SHA-384 and SHA-512 comes in 2 points:
\begin{enumerate}
\item SHA-384 uses different initial values.
\item In the last round of SHA-384 the returned hash value will be  cut to 384 bits.
\end{enumerate}
Since this hash function uses internally a 512-bit hash value, it's
not compatible to the general API in
\texttt{Crypto.Symmetric.Hashfunction}. Hense this hash function
should be avoided using.
%%%%%%%%%%%%%%%%%%%%%%%%%%%%%%%%%%%%%%%%%%%%%%%%%%%%%%%%%%%%%%%%
\subsubsection*{API}
\begin{lstlisting}{}
  -- high level API
  procedure Hash(Message : in Bytes;  Hash_Value : out DW_Block384);
  procedure Hash(Message : in String; Hash_Value : out DW_Block384);
  procedure F_Hash(Filename: in String;Hash_Value : out DW_Block384);
  -- low level API
  procedure Init(Hash_Value     : out    DW_Block512);
  procedure Round(Message_Block : in     DW_Block1024;
                  Hash_Value    : in out DW_Block512);
  function Final_Round
           (Last_Message_Block  : DW_Block1024;
            Last_Message_Length : Message_Block_Length1024;
            Hash_Value          : DW_Block512) return DW_Block384;
\end{lstlisting}
\textbf{Exceptions:}
\begin{itemize}
\item If the message length is greater than
  $2^{128}-1024$:\quad\texttt{SHA2\_Constraint\_Error}.
\item In the procedure \texttt{F\_Hash()}, if the file can not be
  opened:\quad\texttt{File\_Open\_Error}.
\item In the procedure \texttt{F\_Hash()}, if the size of the read
  data is smaller than 0:\quad\texttt{File\_Read\_Error}.
\end{itemize}

%%%%%%%%%%%%%%%%%%%%%%%%%%%%%%%%%%%%%%%%%%%%%%%%%%%%%%%%%%%%%%%%

\subsection{Whirlpool}
Whirlpool is a 512-bit hash function which operates on messages less
than $2^{256}$ bits in length. Whirlpool isn't developed as the SHA-2
hash family by NSA (National Security Agency) but by two free
cryptographers, Vincent Rijmen and Paulo Barreto
\cite{Whirlpool}. Whirlpool is based on a substantially modified AES,
cause Vincent Rijmen is one of the AES (\ref{AES}) designers. If you
want a safer hash function and don't care about compatibility with the
american crypto-standard, then you should consider using Whirlpool.
\subsubsection*{API}
\begin{lstlisting}{}
  -- high level API
  procedure Hash(Message : in Bytes;  Hash_Value : out DW_Block512);
  procedure Hash(Message : in String; Hash_Value : out DW_Block512);
  procedure F_Hash(Filename   :in  String;
                   Hash_Value :out DW_Block512);
  -- low level API
  procedure Init(Hash_Value     : out    DW_Block512);
  procedure Round(Message_Block : in     DW_Block512;
                  Hash_Value    : in out DW_Block512);
  function Final_Round
           (Last_Message_Block : DW_Block512;
            Last_Message_Length: Message_Block_Length512;
            Hash_Value         : DW_Block512) return DW_Block512;
\end{lstlisting}
\textbf{Exceptions:}
\begin{itemize}
\item If the message length is greater than $2^{256}-1024$:\quad
  \texttt{Whirlpool\_Constraint\_Error}.
\item In the procedure \texttt{F\_Hash()}, if the file can not be
  opened:\quad\texttt{File\_Open\_Error}.
\item In the procedure \texttt{F\_Hash()}, if the size of the read
  data is smaller than 0:\quad\texttt{File\_Read\_Error}.
\end{itemize}
